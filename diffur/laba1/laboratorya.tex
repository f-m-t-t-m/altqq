\documentclass[a4paper, 14pt, titlepage, fleqn]{extarticle}
\usepackage[russian]{babel}
\usepackage{amsmath}
\usepackage{listings}
\usepackage{csquotes}
\usepackage{xcolor}


\DeclareMathOperator\artanh{artanh}

\lstdefinestyle{mystyle}{
    basicstyle={\small\ttfamily},
    commentstyle=\color{green},
    keywordstyle=\color{blue},
    stringstyle=\color{red},
    basicstyle=\ttfamily\footnotesize,
    breakatwhitespace=false,         
    breaklines=true,                 
    captionpos=b,                    
    keepspaces=true,                 
    numbers=none,                    
    numbersep=5pt,                  
    showspaces=false,                
    showstringspaces=false,
    showtabs=false,                  
    tabsize=2,
    frame=single,
    aboveskip=3mm,
    belowskip=3mm,
}
\lstset{style=mystyle}

\author {
	Группа Б9120-01.03.02миопд\\
	Агличеев Александр
}
\title {
	Отчёт по лабораторной работе №1
}
\date {
	\today
}

\begin{document}
	\maketitle
	\tableofcontents
	\pagebreak	

	\section*{Введение}
	\addcontentsline{toc}{section}{Введение}
		В данной лабораторной работе мне нужно вычислить неопределенный интеграл, решить численно четыремя методами определенный интеграл на языке \textquote{Go} и найти общее решение дифференциальных уравнение с помощью программ компьютерной математики.
	
	\pagebreak
	\section*{Задание 1}
	\addcontentsline{toc}{section}{Задание 1}
		\subsection*{Постановка задачи}
		\addcontentsline{toc}{subsection}{Постановка задачи}
			\noindent Найти следующий интеграл с подробным описанием всех действий:
			\[\int \frac {\ln{(x^2+1)}} {x^2} \ dx\]
	
		\subsection*{Решение}
		\addcontentsline{toc}{subsection}{Решение}
			\begin{gather*}
				\int \frac {\ln{(x^2+1)}} {x^2} \ dx = 
				(*)\begin{vmatrix}
					u = \ln{(x^2+1)} & du = \dfrac{2x}{x^2+1} \\
					dv = \dfrac{dx}{x^2} & v = -\dfrac{1}{x}
				\end{vmatrix}(*)
				= -\dfrac{\ln{(x^2+1)}}{x} + \\ 
				+ 2\int \dfrac{dx}{x^2+1} =  2\arctg{x} - \dfrac{\ln{(x^2+1)}}{x} + C
			\end{gather*}\\
	
			\textit{Ответ:} \( \displaystyle \int \frac {\ln{(x^2+1)}} {x^2} \ dx =2\arctg{x} - \frac{\ln{(x^2+1)}}{x} + C \)

	\pagebreak
	\section*{Задание 2}
	\addcontentsline{toc}{section}{Задание 2}
		\subsection*{Постановка задачи}
		\addcontentsline{toc}{subsection}{Постановка задачи}
			\noindent Четыремя методами численно вычислить следующий интеграл с точностью \(\varepsilon = 10^{-4}\) . Реализацию решения проводить на языке \textquote{Go}:
			\[\int_{1}^{\infty} \dfrac{\sin{\lfloor x \rfloor}}{\lfloor x \rfloor} dx\]
		\subsection*{Решение}
		\addcontentsline{toc}{subsection}{Решение}
			\begin{enumerate}
				\item Метод левых  прямоугольников при \(n=15041\):\\
					Формула вычисления:
					\[ \int_{1}^{\infty} \dfrac{\sin{\lfloor x \rfloor}}{\lfloor x \rfloor} dx \approx \sum _{i=0} ^{n-1} f(x_i) \cdot \Delta x\]
					Вычисленное значение:
					\[ \int_{1}^{\infty} \dfrac{\sin{\lfloor x \rfloor}}{\lfloor x \rfloor} dx \approx 1.070698, \; \Delta = 0.000098\]
				\item Метод правых  прямоугольников  при \(n=15042\):\\
					Формула вычисления:
					\[ \int_{1}^{\infty} \dfrac{\sin{\lfloor x \rfloor}}{\lfloor x \rfloor} dx \approx \sum _{i=1} ^{n} f(x_i) \cdot \Delta x\]
					Вычисленное значение:
					\[ \int_{1}^{\infty} \dfrac{\sin{\lfloor x \rfloor}}{\lfloor x \rfloor} dx \approx 1.070704,  \; \Delta = 0.000006 \]
				\pagebreak
				\item Метод средних  прямоугольников  при \(n=15042\):\\
					Формула вычисления:
					\[ \int_{1}^{\infty} \dfrac{\sin{\lfloor x \rfloor}}{\lfloor x \rfloor} dx \approx \sum _{i=0} ^{n-1} f\bigg(\dfrac{x_i + x_{i+1}}{2}\bigg) \cdot \Delta x\]
					Вычисленное значение:
					\[ \int_{1}^{\infty} \dfrac{\sin{\lfloor x \rfloor}}{\lfloor x \rfloor} dx \approx 1.070704,  \; \Delta = 0.000006 \]
				\item Метод трапеций  при \(n=15042\):\\
					Формула вычисления:
					\[ \int_{1}^{\infty} \dfrac{\sin{\lfloor x \rfloor}}{\lfloor x \rfloor} dx \approx \sum _{i=0} ^{n-1} \dfrac {f(x_i) + f(x_{i+1})}{2} \cdot \Delta x\]
					Вычисленное значение:	
					\[ \int_{1}^{\infty} \dfrac{\sin{\lfloor x \rfloor}}{\lfloor x \rfloor} dx \approx 1.070724,  \; \Delta =0.000026 \]
				\item Точное решение:
					\[ \int_{1}^{\infty} \dfrac{\sin{\lfloor x \rfloor}}{\lfloor x \rfloor} dx = \dfrac{1}{2}(\pi - 1)  \approx 1.070796 \]
			\end{enumerate}
		\pagebreak
		\subsection*{Код программ}
		\addcontentsline{toc}{subsection}{Код программ}
		\begin{enumerate}
			\item Метод левых прямоугольников:			
				\begin{lstlisting}[language=Go]
package main

import (
	"fmt"
	"math"
)

func integral(x float64) float64 {
    var y = math.Sin(math.Floor(x)) / math.Floor(x)
    return y
}

func checkPrecision(res float64, ans float64) bool {
    if math.Abs(res - ans) <=  0.0001 {
        return true
    }
    return false
}

func leftRectangles(deltaX float64, ans float64) float64 {
    var x float64 = 1
    var res float64 = 0
    var count = 0
    for {
        count++
        var y = integral(x)
        res += y * deltaX
        x += deltaX
        if checkPrecision(res, ans) {
            break
        }
    }
    fmt.Println(count)
    return res
}


func main() {
    const ans float64 = 1.070796
    const delta float64 = 0.0001 
    fmt.Println(leftRectangles(delta, ans))
}
				\end{lstlisting}

			\pagebreak
			\item Метод правых прямоугольников:
				\begin{lstlisting}[language=Go]
package main

import (
	"fmt"
	"math"
)

func integral(x float64) float64 {
    var y = math.Sin(math.Floor(x)) / math.Floor(x)
    return y
}

func checkPrecision(res float64, ans float64) bool {
    if math.Abs(res - ans) <=  0.0001 {
        return true
    }
    return false
}

func rightRectangles(deltaX float64, ans float64) float64 {
    var x float64 = 1
    var res float64 = 0
    var count = 0
    for {
        count++
        x += deltaX
        var y = integral(x)
        res += y * deltaX
        if checkPrecision(res, ans) {
            break
        }
    }
    fmt.Println(count)
    return res
}


func main() {
    const ans float64 = 1.070796
    const delta float64 = 0.0001 
    fmt.Println(rightRectangles(delta, ans))
}
				\end{lstlisting}

			\pagebreak
			\item Метод средних прямоугольников:
				\begin{lstlisting}[language=Go]
package main

import (
	"fmt"
	"math"
)

func integral(x float64) float64 {
    var y = math.Sin(math.Floor(x)) / math.Floor(x)
    return y
}

func checkPrecision(res float64, ans float64) bool {
    if math.Abs(res - ans) <=  0.0001 {
        return true
    }
    return false
}

func midRectangles(deltaX float64, ans float64) float64 {
    var x float64 = 1
    var res float64 = 0
    var count = 0
    for {
        count++
        var y = integral((x + x + deltaX) / 2)
        res += y * deltaX
        x += deltaX
        if checkPrecision(res, ans) {
            break
        }
    }
    fmt.Println(count)
    return res
}


func main() {
    const ans float64 = 1.070796
    const delta float64 = 0.0001 
    fmt.Println(midRectangles(delta, ans))
}
				\end{lstlisting}

			\pagebreak
			\item Метод трапеций:
				\begin{lstlisting}[language=Go]
package main

import (
	"fmt"
	"math"
)

func integral(x float64) float64 {
    var y = math.Sin(math.Floor(x)) / math.Floor(x)
    return y
}

func checkPrecision(res float64, ans float64) bool {
    if math.Abs(res - ans) <=  0.0001 {
        return true
    }
    return false
}

func trapezoids(deltaX float64, ans float64) float64 {
    var x float64 = 1
    var res float64 = 0
    var count = 0
    for {
        count++
        var y = (integral(x) + integral(x + deltaX)) / 2
        x += deltaX
        res += y * deltaX
        if checkPrecision(res, ans) {
            break
        }
    }
    fmt.Println(count)
    return res
}


func main() {
    const ans float64 = 1.070796
    const delta float64 = 0.0001
    fmt.Println(trapezoids(delta, ans))
}
				\end{lstlisting}
		\end{enumerate}

	\pagebreak
	\section*{Задание 3}
	\addcontentsline{toc}{section}{Задание 3}
		\subsection*{Постановка задачи}
		\addcontentsline{toc}{subsection}{Постановка задачи}
		\noindent Для следующих дифференциальных уравнений определить тип и найти общее
		решение с помощью программ компьютерной математики:
		\begin{enumerate}
			\item \( xy' = \dfrac{\sec{xy}}{y} - y \)
			\item \( -(12x + 4y)y' =  x - 8y - 3\)
			\item \( y' = \dfrac{\csc{y}}{y + x \cdot \sec{y}} \)
			\item \( y' = \dfrac{\cos^2{y} \cdot \cos{x}}{\sin^2{x} - 1} \)
		\end{enumerate}
	
	
		\subsection*{Решение}
		\addcontentsline{toc}{subsection}{Решение}
			\noindent Поиск решения будет проводиться в системе компьютерной математики Wolfram Mathematica.
			\begin{enumerate}
				\item \( xy' = \dfrac{\sec{xy}}{y} - y \)

					\textit{Тип уравнения:} Однородное уравнение
		
					\textit{Ответ:}
					\( 2\cos{xy} + 2xy \cdot \sin{xy} - x^2 = C \)
		
				\item \( -(12x + 4y)y' =  x - 8y - 3\)
		
					\textit{Тип уравнения:} Уравнение вида \( y' = f\bigg(\dfrac{a_1x+b_1y+c_1}{ax+by+c}\bigg) \), приводящееся к однородному
		
					\textit{Ответ:} \( \dfrac{25x-3}{10y+5x+3}-\ln{(10y+5x+3)} = C\)
				\pagebreak
				\item \( y' = \dfrac{\csc{y}}{y + x \cdot \sec{y}} \)
		
					\textit{Тип уравнения:} Линейное уравнение по переменной \( x \)
		
					\textit{Ответ:} \( x = C\sec{y} + \dfrac{1}{8}\sec{y}\cdot(\sin{2y} - 2y\cos{2y}) \)
		
				\item \( y' = \dfrac{\cos^2{y} \cdot \cos{x}}{\sin^2{x} - 1} \)
		
					\textit{Тип уравнения:} Уравнение с разделяющимся переменными
	
					\textit{Ответ:} \( \tg{y} = C - \artanh{(\sin{x})} \)
			\end{enumerate}



	\section*{Заключение}
	\addcontentsline{toc}{section}{Заключение}
		\noindent Я решил неопределенный интеграл, написал программу для численного решения интегралов на языке программирования \textquote{Go} и пользовался Wolfram Mathematica для решения дифференциальных уравнений. Оформлял отчёт по работе  в \textquote{TeX Live}.

\end{document}	
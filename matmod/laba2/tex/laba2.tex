\documentclass[a4paper, 14pt]{extarticle}
\usepackage{enumitem}
\usepackage{fefutitle}
\usepackage{listings}
\usepackage{xcolor}
\usepackage{amsmath}
\usepackage{graphicx}
\usepackage[justification=centering]{caption}
\usepackage{float}

\lstdefinestyle{mystyle}{
	basicstyle={\small\ttfamily},
	keywordstyle=\color{orange},
	stringstyle=\color{green},
	basicstyle=\ttfamily\footnotesize,
	breakatwhitespace=false,         
	breaklines=true,                 
	captionpos=b,                    
	keepspaces=true,                 
	numbers=none,                    
	numbersep=5pt,                  
	showspaces=false,                
	showstringspaces=false,
	showtabs=false,                  
	tabsize=2,
	aboveskip=3mm,
	belowskip=3mm,
}
\lstset{style=mystyle}

\begin{document}
	\fefutitle{2}
	\pagebreak	

	\section{Определение цели}
		В данной лабораторной необходимо создать модель электрического утюга с терморегулятором и без него.
		Проанализировать полученную модуль.

	\section{Информация об объекте}
		Утюг - нагревающийся металлический прибор. Работа утюга с электрическим нагревом основана на выделении тепловой энергии при прохождении электрического тока через нагревательный элемент. Температура нагревательного элемента сообщается подошве утюга, которая также нагревается.
		
		Температура нагрева задается отдельным терморегулятором, главная функция которого заключается в своевременном отключении подачи электроэнергии в соответствии с заданным режимом.

	\section{Создание математической модели}
		Введем характеристики, которые необходимы для создания модели:
		\begin{enumerate}[leftmargin=3\parindent, itemsep=0mm]
			\item m - масса подошвы утюга
			\item c - удельная теплоёмкость подошвы утюга
			\item S - площать поверхности подошвы
			\item P - электрическая мощность утюга
			\item T - температура утюга
			\item $T_a$ - температура окружающей среды
		\end{enumerate}
	
		Количество теплоты(\(\Delta Q\)), которое получает тело при увеличении его температуры на величину 
		\( \Delta T = T_2 - T_1 \) вычисляется так:
		\[ \Delta Q = cm \Delta T \], где \( T_1, T_2\) - температуры тела до нагрева и после.
		
		Количество теплоты, отдаваемое утюгом в окружающую среду вычисляется по закону Ньютона-Рихмана:
		\[ Q = kS(T-T_a) \Delta t \]
		, где \(k\) - коэффициент теплообмена
			
		Нагретые тела излучают энергию в виде электромагнитных волн различной длины. Тепловое излучение:
		\[ E = \sigma T^4 \]
		, где \( \sigma = 5.67 \cdot 10^{-8} \) - постоянная Стефана-Больцмана
		
		Составим уравнение теплового баланса:
		\[ \Delta Q = P\Delta t - kS(T-T_a) \Delta t - S \sigma T^4 \Delta t +  S \sigma T_a^4 \Delta t\]
		\[ cm \Delta T = P\Delta t - kS(T-T_a) \Delta t - S \sigma (T^4 - T_a) \Delta t \]
		
		Поделим на \( \Delta t\) и умножим на \(cm\)
		\[ \dfrac{\Delta T}{\Delta t} 
		= \dfrac{P - kS(T-T_a) - S \sigma T^4 +  S \sigma T_a^4}{cm}\]
		
		Перейдём к дифференциальному уравнению и добавим начальное условие(в начальный момент времени
		температура подошвы утюга равна температуре окружающей среды):
		\[
			\begin{cases}
				\dfrac{\Delta T}{\Delta t} = \dfrac{P - kS(T-T_a) - S \sigma (T^4 - T_a) \Delta t}{cm},\\
				T(0) = T_a.
			\end{cases}
		\]
		
		Для утюга с терморегулятором добавим функцию, которая будет отвечать за включение и отключение утюга,
		учитывая заданную максимальную температуру \( T_{max} \)
		\[ H(T) = 
			\begin{cases}
				0, \text{если} (T_{max} - T) <= 5\\
				1, \text{если} (T_{max} - T) > 5
			\end{cases} 
		\]
		
		Тогда дифференциальное уравнение примет следующий вид:
		\[
		\begin{cases}
			\dfrac{\Delta T}{\Delta t} = \dfrac{ P\cdot H(T) - kS(T-T_a) - S \sigma (T^4 - T_a) \Delta t}{cm},\\
			T(0) = T_a.
		\end{cases}
		\]
	
	\section{Анализ модели}
	
	\section{Вывод}
		Таким образом, построена математическая модель утюга с терморегулятором и без него. Она позволяет получить график температур от времени для утюгов с различными площадями подошвы, теплопроводностями и мощностями. 
		
\end{document}	
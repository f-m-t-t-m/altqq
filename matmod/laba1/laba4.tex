\documentclass[a4paper, 14pt]{extarticle}
\usepackage{enumitem}
\usepackage{fefutitle}
\usepackage{amsmath}

\begin{document}
	\fefutitle{1}
	\pagebreak	

	\section{Определение цели}
		В данной лабораторной необходимо создать простую математическую и рассчитать для неё:
		\begin{enumerate}[leftmargin=3\parindent, itemsep=0mm]
			\item минимальную мощность для езды в городе Владивосток
			\item мощность, для разгона до \(100 \text{км}/\text{ч} \)  за 4 секунды
		\end{enumerate}

	\section{Информация об объекте}
		Город Владивосток имеет сложный рельеф: он расположен на холмах. Поэтому в модели необходимо рассматривать движение
		автомобиля по наклонной поверхности.
		
		\(\alpha \) - угол наклона, в среднем составляет \( 15 ^\circ \)

		\( m \) - масса автомобиля. Зависит от типа кузова автомобиля:
		\begin{enumerate}[leftmargin=3\parindent, itemsep=0mm]
			\item Легковой автомобиль - 1000кг
			\item Внедорожник - 2000кг
			\item Пассажирский автобус - 8000кг
		\end{enumerate}

		\( \upsilon \) - скорость автомобиля, варьируется от \(40 \text{км}/\text{ч} \) до \(60 \text{км}/\text{ч} \) 
		
		\( P \) - мощность автомобиля

	\section{Создание математической модели}
		\subsection{Движение под углом}
			Рассмотрим движение автомобиля по наклонной плоскости:
			\( S \) - длина проекции плоскости, по которой движется автомобиль
			\[ S = \upsilon \Delta t \tag{3.1.1} \label{eq:special} \]
			, где \( \Delta t \) - время, за которое автомобиль проезжает расстояние S
	
			Высота, на которую поднимается автомобиль, вычисляется по формуле
			\[ h = S \cdot \sin{\alpha} \tag{3.1.2} \label{eq:special}\]
			Подставим \( (3.1.1) \) в (3.1.2):
			\[ h = \upsilon \Delta t \cdot sin{\alpha}  \tag{3.1.3} \label{eq:special} \]
			Мощность найдём из закона сохранения энергии, не учитывая силу трения
			\[ P \Delta t = mgh  \tag{3.1.4} \label{eq:special}\]
			Выразим \( \Delta t\) из (3.1.3)
			\[ \Delta t = \dfrac{h}{\upsilon \sin{\alpha}} \tag{3.1.5} \label{eq:special} \]
			и подставим в (3.1.4) и получим
			\[ P = mg \upsilon \sin{\alpha} \tag{3.1.6} \label{eq:special} \]
		\subsection{Движение по прямой}
			Для вычисления мощности при движении по прямой воспользуемся
			законом сохранения энергии
			\[ P \Delta t = \dfrac{m\upsilon^2}{2} \tag{3.2.1} \label{eq:special} \]
			Выразим P
			\[ P = \dfrac{m\upsilon^2}{2\Delta t} \tag{3.2.2} \label{eq:special} \]

	\section{Анализ модели}
		
			

	
\end{document}	
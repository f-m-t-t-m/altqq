\documentclass[a4paper, 14pt]{extarticle}
\usepackage{enumitem}
\usepackage{fefutitle}
\usepackage{xcolor}
\usepackage{amsmath}
\usepackage{graphicx}
\usepackage[justification=centering]{caption}
\usepackage{float}

\begin{document}
	\fefutitle{3}
	\pagebreak	

	\section{Определение цели}
		В данной лабораторной необходимо создать модель Лотки-Вольтерра - модель взаимодействия двух видов типа - <<хищник - жертва>>.
		Проанализировать полученную модуль.

	\section{Создание математической модели}
		Рассматривается закрытый ареал, в котором обитают два вида: травоядные("жертва") и хищники. Предполагается, что животные не иммигрируют и не эмигрируют, и
		что еды для травоядных имеется с избытком. Тогда уравнение изменения количества жертв принимает вид:
			\[ \dfrac{dx}{dt} = \alpha x,\]
		где $\alpha$ - коэффициент рождаемости жертв, $x$ - величина популяции жертв, $\dfrac{dx}{dt}$ - скорость прироста популяции жертв

		Пока хищники не охотятся, они вымирают, следовательно, уравнение для численности хищников(без учета численности жертв) принимает вид:
			\[ \dfrac{dy}{dt} = -\gamma y,\]
		где $\gamma$ - коэффициент убыли хищников, $x$ - величина популяции хищников, $\dfrac{dx}{dt}$ - скорость прироста популяции хищников

		При встречах хищников и жертв(частота которых прямо пропорциональна величине $xy$) происходит убийство с коэффициентом $\beta$, сытые хищники способны к воспроизводству с коэффициентом $\delta$.
		С учетом этого, система уравнений модели такова:
		\[ \begin{cases}
			\dfrac{dx}{dt} = \alpha x - \beta xy = (\alpha - \beta y)x, \\
			\dfrac{dy}{dt} = - \gamma y - \delta xy = (\beta x - \gamma)y. \\
		    \end{cases}
		 \]
	
	\section{Анализ модели}

	\section{Вывод}
		Таким образом, построена математическая модель утюга с терморегулятором и без него. Она позволяет получить график температур от времени для утюгов с различными площадями подошвы, теплопроводностями и мощностями. 
		
\end{document}	
\documentclass[14pt, a4paper]{extarticle}
\usepackage{style}
\usepackage{array}
\usepackage{biblatex}
\usepackage{amsmath}
\usepackage{indentfirst}
\usepackage{listings}
\usepackage{xcolor}
\usepackage{fefutitle}

\lstdefinestyle{mystyle}{
	basicstyle={\small\ttfamily},
	keywordstyle=\color{orange},
	stringstyle=\color{green},
	basicstyle=\ttfamily\footnotesize,
	breakatwhitespace=false,         
	breaklines=true,                 
	captionpos=b,                    
	keepspaces=true,                 
	numbers=none,                    
	numbersep=5pt,                  
	showspaces=false,                
	showstringspaces=false,
	showtabs=false,                  
	tabsize=2,
	aboveskip=3mm,
	belowskip=3mm,
	frame=single
}
\lstset{style=mystyle}
\newcolumntype{P}[1]{>{\centering\arraybackslash}p{#1}}

\begin{document}
	
	\fefutitle{2}{Методы Гаусса решения систем линейных алгебраических уравнений}
	\pagebreak
	
	\pagebreak
	
	\section{Введение}
	
	Объектом исследования являются численные методы решения задач линейной алгебры, а также программное обеспечение, реализующее эти методы.
	
	Цель работы – ознакомиться с численными методами решения систем линейных алгебраических уравнений, нахождения обратных матриц, решения проблемы собственных значений, решить предложенные типовые задачи, сформулировать выводы по полученным решениям, отметить достоинства и недостатки методов, сравнить удобство использования и эффективность работы каждой использованной программы, приобрести практические навыки и компетенции, а также опыт самостоятельной профессиональной деятельности, а именно:
	
	\begin{itemize}
		\item создать алгоритм решения поставленной задачи и реализовать его, протестировать программы;
		
		\item освоить теорию вычислительного эксперимента; современных компьютерных технологий;
		
		\item приобрести навыки представления итогов проделанной работы в виде отчета, оформленного в соответствии с имеющимися требованиями, с привлечением современных средств редактирования и печати.
		
	\end{itemize}

	Работа над курсовым проектом предполагает выполнение следующих задач:
	
	\begin{itemize}
		\item дальнейшее углубление теоретических знаний обучающихся и их систематизацию;
		
		\item получение и развитие прикладных умений и практических навыков по направлению подготовки;
		
		\item овладение методикой решения конкретных задач;
		
		\item развитие навыков самостоятельной работы;
		
		\item развитие навыков обработки полученных результатов, анализа и осмысления их с учетом имеющихся литературных данных;
		
		\item приобретение навыков оформления описаний программного продукта;
		
		\item повышение общей и профессиональной эрудиции.
		
	\end{itemize}

	Изученный студентом в ходе работы материал должен способствовать повышению его качества знаний, закреплению полученных навыков и уверенности в выборе путей будущего развития своих профессиональных способностей. 
	
	\section{Основная часть}
	\subsection{Постановка задачи}
	
	Система линейных алгебраических уравнений --- это система линейных алгебраических уравнений, которая записывается в виде:
	\begin{equation}\label{SOLE}
		\begin{cases}
			a_1{}_1 x_1 + a_1{}_2 x_2 + \ldots + a_1{}_n x_n = f_1 \\
			a_2{}_1 x_1 + a_2{}_2 x_2 + \ldots + a_2{}_n x_n = f_2 \\
			\ldots \\
			a_n{}_1 x_1 + a_n{}_2 x_2 + \ldots + a_n{}_n x_n = f_n. \\
		\end{cases}
	\end{equation}
	
	СЛАУ можно представить в матричной форме:
	\begin{equation}\label{SOLEM}
		A x = F,
	\end{equation}
	где $A$ --- матрица, образованная коэффицентами при неизвестных, $x$ --- вектор-столбец переменных, $F$ --- столбец свободных членов.
	
	Из линейной алгебры известно, что решение \eqref{SOLEM} существует и единственно, если определитель матрицы $A$ отличен от нуля. В данной курсовой работе это решение будем искать при помощью метода Гаусса и его модификаций(выбор ведущего элемента по строке, столбцу, всей матрице).
	
	\subsection{Описание алгоритма решения задачи}
		Алгоритм разделяется на два этапа:
		\begin{enumerate}
			\item Прямой ход
			
				Путем элементарных преобразований приводят к ступенчатому виду.
				
				Рассмотрим пример:
				\[ 
				\begin{cases}
					a_{11} \cdot x_1 + a_{12} \cdot x_2 + \ldots + a_{1n} \cdot x_n = b_1\\
					a_{21} \cdot x_1 + a_{22} \cdot x_2 + \ldots + a_{2n} \cdot x_n = b_2\\
					\ldots\\
					a_{n1} \cdot x_1 + a_{n2} \cdot x_2 + \ldots + a_{nn} \cdot x_n = b_n\\
				\end{cases}	
				\]
				
				Делим первую строку на $a_{11}$ и прибавляем первую строку к остальным с такими коэффициентами, чтобы их коэффициенты в первом столбце обнулились -- очевидно, что при прибавлении к первой строке необходимо умножить на $-a_{i1}$.
				\[ 
				\begin{cases}
					a_{11} \cdot x_1 +&a_{12} \cdot x_2 + \ldots + a_{1n} \cdot x_n = b_1\\
					&a'_{22} \cdot x_2 + \ldots + a'_{2n} \cdot x_n = b'_2\\
					&\ldots\\
					&a'_{n2} \cdot x_2 + \ldots + a'_{nn} \cdot x_n = b'_n\\
				\end{cases}	
				\]
				
				Проделываем те же операции и с другими строками и получим:
				\begin{align*}\begin{cases}
						a_{11} \cdot x_1 +a_{12} \cdot x_2 + \ldots + &a_{1n} \cdot x_n = b_1\\
						\hfill a'_{22} \cdot x_2 + \ldots + &a'_{2n} \cdot x_n = b'_2\\
						& \ldots\\
						\hfill &a^{(n-1)}_{nn} \cdot x_n = b^{(n-1)}_n\\
				\end{cases}\end{align*}	
			\item Обратных ход
			
				Начиная с последнего уравнения, выражаем переменную на главной диагонали и подставляем вычисленные решения, и так далее, <<поднимаясь вверх>>. 
				\[ x_i = \dfrac{1}{a_{ii}} \Bigg(b_i - \sum_{j=i+1}^{n} a_{ij}\cdot x_j\Bigg) \]
		\end{enumerate}
	
		Описанный выше алгоритм работает, только если $a_{ii} \neq 0$. Чтобы алгоритм работал в таком случае необходимо выбрать ведущий элемент и переставить строки/столбцы так, чтобы $a_{ii}$ стал ненулевым элементом. В качестве ведущего элемента стоит выбирать наибольший по модулю, чтобы вычислительная погрешность медленнее накапливалась.
	
	\subsection{Описание тестов, использованных для отладки}
		Для тестов и отладки использовались следующие СЛАУ:
		\begin{enumerate}
			\item $ \begin{cases}
						2x_1 + x_2 - x_3 = 8\\
						-3x_1 - x_2 + 2x_3 = 11\\
						-2x_1 + x_2 - x_3 = -3
					\end{cases}$ 
				\begin{center}
					\begin{tabular}{ |m{10em}|c|c|c| }
						\hline
						Метод & $x_1$& $x_2$& $x_3$\\
						\hline
						Гаусс & 2 & 3 & -1\\
						\hline
						Гаусс с выбором ведущего элемента в строке  & 2 & 3 & -1\\
						\hline
						Гаусс с выбором ведущего элемента в столбце  & 2 & 3 & -1\\
						\hline
						Гаусс с выбором ведущего элемента в матрице  & 2 & 3 & -1\\
						\hline
					\end{tabular}
				\end{center}
			\item $ \begin{cases}
						0.183x_1 + 0.081x_2 + 0.521x_3 + 0.498x_4 = 0.263\\
						0.887x_1 + 5.526x_2 + 0.305x_3 + 0.037x_4 = 0.744\\
						0.678x_1 + 0.658x_2 + 2.453x_3 + 0.192x_4 = 0.245\\
						4.957x_1 + 0.398x_2 + 0.232x_3 + 0.567x_4 = 0.343
					\end{cases}$
				\begin{center}
					\begin{tabular}{ |m{10em}|c|c|c|c| }
						\hline
						Метод & $x_1$& $x_2$& $x_3$& $x_4$\\
						\hline
						Гаусс & 0,051614 & 0,128859 & 0,047621 & 0,043766\\
						\hline
						Гаусс с выбором ведущего элемента в строке  & 0,051614 & 0,128859 & 0,047621 & 0,043766\\
						\hline
						Гаусс с выбором ведущего элемента в столбце  & 0,051614 & 0,128859 & 0,047621 & 0,043766\\
						\hline
						Гаусс с выбором ведущего элемента в матрице & 0,051614 & 0,128859 & 0,047621 & 0,043766\\
						\hline
					\end{tabular}
				\end{center}
			\item $ \begin{cases}
				0.183x_1 + 0.081x_2 + 0.521x_3 + 0.498x_4 = 0.263\\
				0.887x_1 + 5.526x_2 + 0.305x_3 + 0.037x_4 = 0.744\\
				0.678x_1 + 0.658x_2 + 2.453x_3 + 0.192x_4 = 0.245\\
				0.678x_1 + 0.658x_2 + 2.453x_3 + 0.192x_4 = 0.245
			\end{cases}$
		
			В третьем тесте не получили ответ, т.к. слау содержит одинаковые строки и имеет бесконечное число решений.
		\end{enumerate}
	
	\pagebreak
	\subsection{Вычислительные эксперименты}
		\begin{enumerate}
			 \item $\begin{cases}
			 			23x_1 + 27x_2 + 23x_3 + 19x_4 + 34x_5 + 20x_6 = 34\\
			 			13x_1 + 15x_2 + 30x_3 + 26x_4 + 22x_5 + 11x_6 = 22\\
			 			12x_1 + 34x_2 + 28x_3 + 11x_4 + 18x_5 + 33x_6 = 27\\
			 			21x_1 + 25x_2 + 23x_3 + 33x_4 + 25x_5 + 21x_6 = 11\\
			 			30x_1 + 10x_2 + 13x_3 + 29x_4 + 19x_5 + 12x_6 = 27\\
			 			11x_1 + 14x_2 + 17x_3 + 26x_4 + 35x_5 + 20x_6 = 17\\
			 		\end{cases}$
		 		
		 		\begin{tabular}{ |P{1em}|P{7em}|P{7em}|P{7em}|P{7em}| }
		 			\hline
		 			-- & Гаусс & Гаусс с выбором ведущего элемента в строке & Гаусс с выбором ведущего элемента в столбце&Гаусс с выбором ведущего элемента в матрице\\
		 			\hline
		 			$x_1$ & 1,541615 &  1,541615 & 1,541615 & 1,541615\\
		 			\hline 
		 			$x_2$ & -1,339768 &  -1,339768 & -1,339768 & -1,339768\\
		 			\hline
		 			$x_3$ & 1,284753 &  1,284753 & 1,284753 & 1,284753\\
		 			\hline
		 			$x_4$ & -1,392177 &  -1,392177 & -1,392177 & -1,392177\\
		 			\hline
		 			$x_5$ & 0,536146 &  0,536146 & 0,536146 & 0,536146\\
		 			\hline
		 			$x_6$ & 0,719484 &  0,719484 & 0,719484 & 0,719484\\
		 			\hline
		 		\end{tabular} 
			 \item $\begin{cases} 
			 			49x_1 + 8x_2 + 31x_3 + 17x_4 + 24x_4 + 22x_6 + 4x_7 + 3x_8 + 39x_9 + 36x_10 = 13\\
			 			10x_1 + 36x_2 + 3x_3 + 5x_4 + 42x_4 + 46x_6 + 36x_7 + 48x_8 + 5x_9 + 26x_10 = 19\\
			 			3x_1 + 46x_2 + 10x_3 + 35x_4 + 50x_4 + 8x_6 + 29x_7 + 37x_8 + 26x_9 + 36x_10 = 44\\
			 			20x_1 + 24x_2 + 39x_3 + 30x_4 + 39x_4 + 27x_6 + 26x_7 + 14x_8 + 41x_9 + 9x_10 = 48\\
			 			44x_1 + 7x_2 + 15x_3 + 43x_4 + 19x_4 + 39x_6 + 16x_7 + 33x_8 + 21x_9 + 27x_10 = 23\\
			 			8x_1 + 4x_2 + 24x_3 + 44x_4 + 35x_4 + 20x_6 + 15x_7 + 5x_8 + 41x_9 + 20x_10 = 45\\
			 			27x_1 + 46x_2 + 31x_3 + 19x_4 + 38x_4 + 19x_6 + 41x_7 + 27x_8 + 24x_9 + 33x_10 = 26\\
			 			30x_1 + 24x_2 + 17x_3 + 23x_4 + 48x_4 + 33x_6 + 31x_7 + 13x_8 + 29x_9 + 4x_10 = 3\\
			 			31x_1 + 23x_2 + 41x_3 + 8x_4 + 44x_4 + 50x_6 + 42x_7 + 45x_8 + 6x_9 + 36x_10 = 41\\
			 			29x_1 + 40x_2 + 43x_3 + 36x_4 + 6x_4 + 42x_6 + 28x_7 + 7x_8 + 11x_9 + 27x_10 = 49
			 		\end{cases}$
		 			\begin{tabular}{ |P{1em}|P{7em}|P{7em}|P{7em}|P{7em}| }
			 			\hline
			 			-- & Гаусс & Гаусс с выбором ведущего элемента в строке & Гаусс с выбором ведущего элемента в столбце&Гаусс с выбором ведущего элемента в матрице\\
			 			\hline
			 			$x_1$ & -0,902286 &  -0,902286 & -0,902286 & -0,902286\\
			 			\hline 
			 			$x_2$ & 0,236589 &  0,236589 & 0,236589 & 0,236589\\
			 			\hline
			 			$x_3$ & 1,261807 &  1,261807 & 1,261807 & 1,261807\\
			 			\hline
			 			$x_4$ & 0,525887 &  0,525887 & 0,525887 & 0,525887\\
			 			\hline
			 			$x_5$ & -0,012908 &  -0,012908 & -0,012908 & -0,012908\\
			 			\hline
			 			$x_6$ & 0,220490 &  0,220490 & 0,220490 & 0,220490\\
			 			\hline
			 			$x_7$ & -0,840477 &  -0,840477 & -0,840477 & -0,840477\\
			 			\hline
			 			$x_8$ & 0,664099 &  0,664099 & 0,664099 & 0,664099\\
			 			\hline
			 			$x_9$ & 0,046808 &  0,046808 & 0,046808 & 0,046808\\
			 			\hline
			 			$x_{10}$ & 0,062953 &  0,062953 & 0,062953 & 0,062953\\
			 			\hline
		 			\end{tabular}
		\end{enumerate}
	\subsection{Оценка количества арифметических операций}
		Будем рассматривать квадратную матрицу размерностью $n$. 
		
		Посчитаем количество операций во время прямого хода. Прибавление $i$-ой строки к следующей требует $n-i$ операций. Тогда количество операций умножения и сложения для преобразования матрицы к ступенчатой равно $\sum_{i=1}^{n-1} (n-i)^2 = \dfrac{1}{6}n(n-1)(2n-1) $, а делений: $\sum_{i=1}^{n-1} (n-i) = \dfrac{1}{2}n(n-1)$.
		
		На обратном ходу нам потребуется $2 \sum_{i=1}^{n-1} (n-i) = n(n-1)$ операций умножения и вычитания и $n$ делений.
		
		Таким образом, в сумме нам нужно $\dfrac{n^3}{2} + \dfrac{4n^2}{3}-\dfrac{5n}{6}$ операций.
	\section{Заключение}
	
	В результате работы над курсовым проектом были реализованы метод Гаусса и его модификации: с выбором главного элемента по строке, столбцу, всей матрице. Методы были протестированы и отлажены на серии тестов, а затем применены для вычислительных эксперентов.
	
	Приобрел практические навыки владения:
	
	\begin{itemize}
	\item современными численными методами решения задач линейной алгебры;
	
	\item основами алгоритмизации для численного решения задач линейной алгебры на языке программирования Python 3;
	
	\item инструментальными средствами, поддерживающими разработку программного обеспечения для численного решения задач линейной алгебры;
	
	\end{itemize}
	а также навыками представления итогов проделанной работы в виде отчета, оформленного в соответствии с имеющимися требованиями, с привлечением современных средств редактирования и печати.
	\section{Список использованных источников}
	\begin{enumerate}
		\item Бахвалов Н.С. Численные методы / Н.С. Бахвалов, Н.П. Жидков, Г.М. Кобельков. -- М.: Наука, 2002 г. – 630 с..
		
		\item Фаддеев Л.К. Вычислительные методы линейной алгебры / Л.К. Фаддеев, В.Н. Фаддеева. -- М.: Физматгиз, 1963. -- 656 с.
	\end{enumerate}
	
	\section{Приложения (тексты программ)}
	Метод Гаусса:		
	\begin{lstlisting}[language=java]
public class GaussLinearEquationSolver {
	
	public double[] solve(double[][] a, double[] b) {
		if (a.length != a[0].length) {
			throw new IllegalArgumentException();
		}
		double[][] f = new double[a.length][];
		for (int i = 0; i < a.length; ++i) {
			f[i] = Arrays.copyOf(a[i], a[i].length);
		}
		double[] s = Arrays.copyOf(b, b.length);
		
		forwardStep(f, s);
		return backwardStep(f, s);
	}
	
	public void forwardStep(double[][] a, double[] b) {
		int lastIndexA = a.length - 1;
		for(int i = 0; i < lastIndexA; ++i) {
			if (a[i][i] == 0) {
				throw new ArithmeticException();
			}
			for(int j = i + 1; j < a.length; ++j) {
				double multiplier = a[j][i]/a[i][i];
				double[] subtractLineA = copyAndMultiplyArray(a[i], multiplier);
				b[j] -= b[i]*multiplier;
				subtractLine(a[j], subtractLineA);
			}
		}
		if (a[lastIndexA][lastIndexA] == 0) {
			throw new ArithmeticException();
		}
	}
	
	public double[] backwardStep(double[][] a, double[] b) {
		double[] ans = Arrays.copyOf(b, b.length);
		int lastIndexA = a.length - 1;
		for(int i = lastIndexA; i >= 0; --i) {
			for(int j = i + 1; j < b.length; ++j) {
				ans[i] -= ans[j]*a[i][j];
			}
			ans[i] /= a[i][i];
		}
		return ans;
	}
}
	\end{lstlisting}

	Метод Гаусса с выбором велущего элемента в столбце:		
\begin{lstlisting}[language=java]
public class GaussLinearEquationSolverColumnPivot extends GaussLinearEquationSolver {
	
	@Override
	public void forwardStep(double[][] a, double[] b) {
		int lastIndexA = a.length - 1;
		for(int i = 0; i < lastIndexA; ++i) {
			swapRows(a, b, getColumnPivot(a, i), i);
			if (a[i][i] == 0) {
				throw new ArithmeticException(String.format();
			}
			for(int j = i + 1; j < a.length; ++j) {
				double multiplier = a[j][i]/a[i][i];
				double[] subtractLineA = copyAndMultiplyArray(a[i], multiplier);
				b[j] -= b[i]*multiplier;
				subtractLine(a[j], subtractLineA);
			}
		}
		if (a[lastIndexA][lastIndexA] == 0) {
			throw new ArithmeticException();
		}
	}
	
	public int getColumnPivot(double[][] a, int colNumber) {
		int idxOfMax = colNumber;
		for (int row = colNumber; row < a.length; ++row) {
			idxOfMax = Math.abs(a[row][colNumber]) > Math.abs(a[idxOfMax][colNumber]) ? row : idxOfMax;
		}
		return idxOfMax;
	}
}
\end{lstlisting}

	Метод Гаусса с выбором велущего элемента в строке:		
\begin{lstlisting}[language=java]
public class GaussLinearEquationSolverStrokePivot extends GaussLinearEquationSolver {
	@Override
	public double[] solve(double[][] a, double[] b) {
		if (a.length != a[0].length) {
			throw new IllegalArgumentException();
		}
		double[][] f = new double[a.length][];
		for (int i = 0; i < a.length; ++i) {
			f[i] = Arrays.copyOf(a[i], a[i].length);
		}
		double[] s = Arrays.copyOf(b, b.length);
		int[] ansConsequence = new int[a[0].length];
		for (int i = 0; i < ansConsequence.length; ++i) {
			ansConsequence[i] = i;
		}
		forwardStep(f, s, ansConsequence);
		return backwardStep(f, s, ansConsequence);
	}
	
	public void forwardStep(double[][] a, double[] b, int[] ansConsequence) {
		int lastIndexA = a.length - 1;
		
		
		for(int i = 0; i < lastIndexA; ++i) {
			swapCols(a, getStrokePivot(a, i), i, ansConsequence);
			if (a[i][i] == 0) {
				throw new ArithmeticException(String.format();
			}
			for(int j = i + 1; j < a.length; ++j) {
				double multiplier = a[j][i]/a[i][i];
				double[] subtractLineA = copyAndMultiplyArray(a[i], multiplier);
				b[j] -= b[i]*multiplier;
				subtractLine(a[j], subtractLineA);
			}
			if (a[lastIndexA][lastIndexA] == 0) {
				throw new ArithmeticException();
			}
		}
	}
	
	public double[] backwardStep(double[][] a, double[] b, int[] ansConsequence) {
		double[] ans =  super.backwardStep(a, b);
		return buildAnswer(ans, ansConsequence);
	}
	
	public int getStrokePivot(double[][] a, int rowNumber) {
		int idxOfMax = rowNumber;
		for (int col = rowNumber; col < a[0].length; ++col) {
			idxOfMax = Math.abs(a[rowNumber][col]) > Math.abs(a[rowNumber][idxOfMax]) ? col : idxOfMax;
		}
		return idxOfMax;
	}
	
	double[] buildAnswer(double[] ans, int[] ansConsequence) {
		double[] res = new double[ans.length];
		for (int i = 0; i < ans.length; ++i) {
			res[ansConsequence[i]] = ans[i];
		}
		return res;
	}
}
\end{lstlisting}

	Метод Гаусса с выбором велущего элемента во всей матрице:		
\begin{lstlisting}[language=java]
public class GaussLinearEquationSolverPivot extends GaussLinearEquationSolver {
	
	public static class Point {
		private final int x;
		private final int y;
		
		public int getX() {
			return x;
		}
		
		public int getY() {
			return y;
		}
		
		public Point(int x, int y) {
			this.x = x;
			this.y = y;
		}
	}
	
	@Override
	public double[] solve(double[][] a, double[] b) {
		if (a.length != a[0].length) {
			throw new IllegalArgumentException();
		}
		double[][] f = new double[a.length][];
		for (int i = 0; i < a.length; ++i) {
			f[i] = Arrays.copyOf(a[i], a[i].length);
		}
		double[] s = Arrays.copyOf(b, b.length);
		int[] ansConsequence = new int[a[0].length];
		for (int i = 0; i < ansConsequence.length; ++i) {
			ansConsequence[i] = i;
		}
		forwardStep(f, s, ansConsequence);
		return backwardStep(f, s, ansConsequence);
	}
	
	public void forwardStep(double[][] a, double[] b, int[] ansConsequence) {
		int lastIndexA = a.length - 1;
		for(int i = 0; i < lastIndexA; ++i) {
			Point max = findPivot(a, i, i);
			swapRows(a, b, max.getX(), i);
			swapCols(a, max.getY(), i, ansConsequence);
			if (a[i][i] == 0) {
				throw new ArithmeticException();
			}
			for(int j = i + 1; j < a.length; ++j) {
				double multiplier = a[j][i]/a[i][i];
				double[] subtractLineA = copyAndMultiplyArray(a[i], multiplier);
				b[j] -= b[i]*multiplier;
				subtractLine(a[j], subtractLineA);
			}
		}
		if (a[lastIndexA][lastIndexA] == 0) {
			throw new ArithmeticException();
		}
	}
	
	public double[] backwardStep(double[][] a, double[] b, int[] ansConsequence) {
		double[] ans =  super.backwardStep(a, b);
		return buildAnswer(ans, ansConsequence);
	}
	
	double[] buildAnswer(double[] ans, int[] ansConsequence) {
		double[] res = new double[ans.length];
		for (int i = 0; i < ans.length; ++i) {
			res[ansConsequence[i]] = ans[i];
		}
		return res;
	}
	
	public Point findPivot(double[][] a, int rowNumber, int colNumber) {
		int colOfMax = colNumber;
		int rowOfMax = rowNumber;
		for (int i = rowNumber; i < a.length; ++i) {
			for (int j = colNumber; j < a[0].length; ++j) {
				if (Math.abs(a[i][j]) > Math.abs(a[rowOfMax][colOfMax])) {
					rowOfMax = i;
					colOfMax = j;
				}
			}
		}
		return new Point(rowOfMax, colOfMax);
	}	
}
\end{lstlisting}
	
	\section{Решение теоретических задач}
		\subsection{Задача 1}
			\subsubsection{Постановка задачи}
			Найдите соотношение эквивалентности, связывающее норму $M(A) = n \times \underset{1 \leq i,j \leq n}{\max{|a_{ij}|}}$ с $\|A\|_{\infty}$. Проверьте эксперементально.
			\subsubsection{Решение}
				\[ \|A\|_{\infty} = \underset{i}{\max} \sum_{j=1}^{n}|a_{ij}| \geq \underset{i,j}{\max}|a_{ij}| = \dfrac{1}{n} M(A)  \]
				\[ \|A\|_{\infty} = \underset{i}{\max} \sum_{j=1}^{n}|a_{ij}| \leq n \times \underset{i,j}{\max}|a_{ij}| = M(A)  \]
				\[ \dfrac{1}{n}M(A) \leq \|A\|_{\infty} \leq M(A) \]
				
				Проверим эксперементально:
				\begin{enumerate}
					\item 
					$\begin{pmatrix}
						1 & 2 & 3\\
						4 & 4 & 4\\
						2 & 1 & 2	
					\end{pmatrix}$
					\[ M(A) = 3 \cdot 4 = 12 \]
					\[ \|A\|_{\infty} = 4+4+4 = 12 \]
					\[ 4 \leq 12 \leq 12 \]
					\item
					$\begin{pmatrix}
						1 & 2 & 3\\
						0 & 0 & 7\\
						2 & 1 & 2	
					\end{pmatrix}$
					\[ M(A) = 3 \cdot 7 = 21 \]
					\[ \|A\|_{\infty} = 7 \]
					\[ 7 \leq 7 \leq 21 \]
				\end{enumerate}
		\subsection{Задача 2}
		\subsubsection{Постановка задачи}
		Докажите теоретически и проверьте эксперементально, что число обусловленности 
		$\mu(A) = \mu(\alpha A))$, где $\alpha$ - число, $\alpha \neq 0$.
		\subsubsection{Решение}
		Формула для вычисления обусловленности: 
		\[ \mu(A) = \|A\|\cdot \|A^{-1}\|\]
		Найдем число обусловленности для $\alpha A$:
		\[ \mu(\alpha A) = \|\alpha A\| \cdot \|(\alpha A)^{-1}\| = \|\alpha A\| \cdot \|\dfrac{1}{\alpha} A^{-1}\| = \alpha \|A\| \cdot \dfrac{1}{\alpha} \|A^{-1}\| = \|A\|\cdot \|A^{-1}\| \]
		Получили равенство $\mu(A) = \mu(\alpha A)$.
		
		Проверим эксперементально:
		\[ A = \begin{pmatrix}
					7 & 7 & 6\\
					14 & 12 & 9\\
					12 & 6 & 14
				\end{pmatrix}
		\alpha A = \begin{pmatrix}
						14 & 14 & 12\\
						28 & 24 & 18\\
						24 & 12 & 28
					\end{pmatrix}, \alpha = 2
		\]
		\[A^{-1} = \begin{pmatrix}
						-0.64044944 & 0.34831461 & 0.0505618\\
						 0.49438202 & -0.14606742 & -0.11797753\\
						 0.33707865 & -0.23595506 &  0.078651695
					\end{pmatrix}
		\]
		\[(\alpha A)^{-1} = \begin{pmatrix}
			-0.32022472 & 0.1741573 & 0.0252809\\
			 0.24719101 & -0.07303371 & -0.05898876\\
			 0.16853933 & -0.11797753 & 0.03932584
		\end{pmatrix}
		\]
		\[ \mu(A) = \|A\|_{\infty} \cdot \|A^{-1}\|_{\infty} = 35 \cdot 1.039326 = 36.376404 \]
		\[ \mu(\alpha A) = \|\alpha A\|_{\infty} \cdot \|(\alpha A)^{-1}\|_{\infty} = 70 \cdot 0.519663 = 36.376404 \]
		\[ \mu(A) = \mu(\alpha A) = 36.376404 \]
\end{document}} 

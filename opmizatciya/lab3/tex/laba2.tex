\documentclass[a4paper, 14pt]{extarticle}
\usepackage{fefutitle}
\usepackage{listings}
\usepackage{xcolor}
\usepackage{amsmath}
\usepackage{amssymb}
\usepackage{changepage}
\usepackage{makecell}
\usepackage{longtable}
\usepackage{graphicx}

\newenvironment{widerequation}{%
	\begin{adjustwidth}{-2cm}{-2cm}\[}
		{\]\end{adjustwidth}}


\lstdefinestyle{mystyle}{
	basicstyle={\small\ttfamily},
	keywordstyle=\color{orange},
	stringstyle=\color{green},
	basicstyle=\ttfamily\footnotesize,
	breakatwhitespace=false,         
	breaklines=true,                 
	captionpos=b,                    
	keepspaces=true,                 
	numbers=none,                    
	numbersep=5pt,                  
	showspaces=false,                
	showstringspaces=false,
	showtabs=false,                  
	tabsize=2,
	aboveskip=3mm,
	belowskip=3mm,
}
\lstset{style=mystyle}
\setcounter{MaxMatrixCols}{21}

\begin{document}
	\fefutitle{2}
	\pagebreak	
	
	\section{Постановка задачи}
		Дана задача:
			\[\begin{cases}
				c \cdot x \rightarrow max\\
				Ax \leq b\\
				x \geq 0
			\end{cases}\]
		Где $c$ -- неотрицательный 6-мерный вектор, $x$ -- неотрицательный
		6-мерный вектор неизвестных, который необходимо найти, $A$ -- матрица $6 \times 8$, $b$ -- неотрицательный 8-мерный вектор
		
		\begin{widerequation}
			A = \begin{pmatrix}
					13 & 27 & 9  & 19 & 6  & 27\\
					8  & 16 & 8  & 5  & 13 & 22\\
					15 & 26 & 6  & 17 & 15 & 10\\
					28 & 22 & 4  & 28 & 4  & 17\\
					14 & 11 & 15 & 8  & 15 & 15\\
					1  & 28 & 20 & 23 & 24 & 25\\
					9  & 17 & 6  & 6  & 13 & 23\\
					15 & 19 & 3  & 29 & 5 & 7
				\end{pmatrix}
			b = \begin{pmatrix}
					6 \\ 5 \\ 2 \\17 \\ 2 \\ 9 \\ 21 \\ 5
				\end{pmatrix}	
			c = \begin{pmatrix}
					7 & 16 & 5 & 11 & 27 & 10
				\end{pmatrix}	 
		\end{widerequation}
	
		Решать будем симплекс-методом. Для начала приведем задачу к
		каноническому виду. Введем дополнительный 8-мерный вектор переменных $z = Ax - b$.
		Тогда к вектору c дописываем 8 нулей и рассматриваем вектор $\begin{pmatrix}x \\ y \end{pmatrix}$. 
		К матрице A справа дописываем единичную матрицу получаем:
		\[
			\begin{cases}
				(c, 0) \cdot \begin{pmatrix} x \\ z \end{pmatrix} \rightarrow max \\
				(AI) \cdot \begin{pmatrix} x \\ z \end{pmatrix} = b\\
				x, z \geq 0
			\end{cases}  
		\]
		\pagebreak
		
		\textbf{Прямая задача}
		
		Составим симплекс-таблицу. Первая строка -- расширенный вектор $c$, где элементы мы запишем со знаком минус, чтобы решать
		задачу на минимум. Остальные строки -- расширенная матрица $A$,
		последний столбик -- вектор $b$, а первый элемент последнего столбца
		- значение целевой функции, равное $0$.
		
		Видим, что в первой строке (не включая значение целевой функции) есть отрицательные элементы, а значит оптимальное решение
		еще не найдено.
		
		\underline{Разрешающая колонка} находится путем выборки такого столбца,
		у которого элемент строки целевой функции отрицательный. Мы будем брать отрицательный элемент, максимальный по модулю.
		
		\underline{Разрешающей строкой} будет строка, содержащая наименьшее \textit{положительное} отношение свободного числа к элементу разрешающего столбца.
		Элемент, расположенный на пересечении разрешающих столбца
		и строки, называется разрешающим элементом.
		\begin{widerequation}\begin{pmatrix}%
			-7.0 & -16.0 &-5.0 & -11.0 & -27.0 & -10.0 & 0.0 & 0.0 & 0.0 & 0.0 & 0.0 & 0.0 & 0.0 & 0.0 & 0.0\\
			13.0 & 27.0  &9.0  & 19.0  & 6.0   & 27.0  & 1.0 & 0.0 & 0.0 & 0.0 & 0.0 & 0.0 & 0.0 & 0.0 & 6.0\\
			8.0  & 16.0  &8.0  & 5.0   & 13.0  & 22.0  & 0.0 & 1.0 & 0.0 & 0.0 & 0.0 & 0.0 & 0.0 & 0.0 & 5.0\\
			15.0 & 26.0  &6.0  & 17.0  & \colorbox{blue!30}{15.0}  & 10.0  & 0.0 & 0.0 & 1.0 & 0.0 & 0.0 & 0.0 & 0.0 & 0.0 & 2.0\\
			28.0 & 22.0  &4.0  & 28.0  & 4.0   & 17.0  & 0.0 & 0.0 & 0.0 & 1.0 & 0.0 & 0.0 & 0.0 & 0.0 & 17.0\\
			14.0 & 11.0  &15.0 & 8.0   & 15.0  & 15.0  & 0.0 & 0.0 & 0.0 & 0.0 & 1.0 & 0.0 & 0.0 & 0.0 & 2.0\\
			1.0  & 28.0  &20.0 & 23.0  & 24.0  & 25.0  & 0.0 & 0.0 & 0.0 & 0.0 & 0.0 & 1.0 & 0.0 & 0.0 & 9.0\\
			9.0  & 17.0  &6.0  & 6.0   & 13.0  & 23.0  & 0.0 & 0.0 & 0.0 & 0.0 & 0.0 & 0.0 & 1.0 & 0.0 & 21.0\\
			15.0 & 19.0  &3.0  & 29.0  & 5.0   & 7.0   & 0.0 & 0.0 & 0.0 & 0.0 & 0.0 & 0.0 & 0.0 & 1.0 & 5.0
		\end{pmatrix}\end{widerequation}
	
		Начальное угловое решение:
		\[ \begin{pmatrix}
			0 & 0 & 0 & 0 & 0 & 0 & 6.0 & 5.0 & 2.0 & 17.0 & 2.0 & 9.0 & 21.0 & 5.0
		\end{pmatrix} \]
	
		Разрешающий столбец = 5
		
		Разрешающая строка = 4
		
		Разрешающий элемент = 15.0
		
       	Преобразовываем строки матрицы, то есть один из базисных столбцов станет \textbf{не} базисным, а разрешающий столбец – базисным:
       	\begin{enumerate}
       		\item Элементы разрешающей строки делим на разрешающий элемент.
       		\item Преобразования остальных строк: Новая строка = Строка -
       		элемент строки в разрешающем столбце * элемент разрешающей строки
       	\end{enumerate}
       
       В первой строке (не включая значение целевой функции) есть отрицательные элементы, а значит оптимальное решение еще не найдено
       	
       	\begin{widerequation}\begin{pmatrix}
       		20.0&30.8&5.8&19.6&0.0&8.0&0.0&0.0&1.8&0.0&0.0&0.0&0.0&0.0&3.6\\
       		7.0&16.6&6.6&12.2&0.0&23.0&1.0&0.0&-0.4&0.0&0.0&0.0&0.0&0.0&5.2\\
       		-5.0&-6.53&2.8&-9.73&0.0&13.33&0.0&1.0&-0.87&0.0&0.0&0.0&0.0&0.0&3.27\\
       		1.0&1.73&0.4&1.13&1.0&0.67&0.0&0.0&0.07&0.0&0.0&0.0&0.0&0.0&0.13\\
       		24.0&15.07&2.4&23.47&0.0&14.33&0.0&0.0&-0.27&1.0&0.0&0.0&0.0&0.0&16.47\\
       		-1.0&-15.0&9.0&-9.0&0.0&5.0&0.0&0.0&-1.0&0.0&1.0&0.0&0.0&0.0&0.0\\
       		-23.0&-13.6&10.4&-4.2&0.0&9.0&0.0&0.0&-1.6&0.0&0.0&1.0&0.0&0.0&5.8\\
       		-4.0&-5.53&0.8&-8.73&0.0&14.33&0.0&0.0&-0.87&0.0&0.0&0.0&1.0&0.0&19.27\\
       		10.0&10.33&1.0&23.33&0.0&3.67&0.0&0.0&-0.33&0.0&0.0&0.0&0.0&1.0&4.33
       	\end{pmatrix}\end{widerequation}
		             
  		В первой строке (не включая значение целевой функции) \textbf{НЕТ}
  		отрицательных элементов, а значит оптимальное решение найдено.
  		
  		\underline{Оптимальное решение:}
  		$ \begin{pmatrix} 0 & 0 & 0 & 0 & 0.13 & 0 \end{pmatrix} $
  		
  		\underline{Целевая функция}: $3.6$
  		
  		\textbf{Двойственная задача}
  		
  		Двойственная задача будет иметь вид:
  		\[ 
  			\begin{cases} 
				b \cdot x \rightarrow min\\
				A^Ty \geq c\\
				y \geq 0	  		
  			\end{cases} 
  		\]
  		Где $c$ -- неотрицательный 6-мерный вектор, $x$ -- неотрицательный
  		8-мерный вектор неизвестных, который необходимо найти, $A^T$ -- матрица $8 \times 6$, $b$ -- неотрицательный 8-мерный вектор
  		\begin{widerequation}
  			A^T = \begin{pmatrix}
  				13&8&15&28&14&1.0&9&15\\
  				27&16&26&22&11&28&17&19\\
  				9&8&6.0&4&15&20&6&3\\
  				19&5&17&28&8&23&6&29\\
  				6&13&15&4&15&24&13&5\\
  				27&22&10&17&15&25&23&7
  			\end{pmatrix}
  			b = \begin{pmatrix}
  				6 & 5 & 2 & 17 & 2 & 9 & 21 & 5
  			\end{pmatrix}	
  			c = \begin{pmatrix}
  				7 \\ 16 \\ 5 \\ 11 \\ 27 \\ 10
  			\end{pmatrix}	 
  		\end{widerequation}
 		Для начала приведем задачу к каноническому виду. Введем дополнительный 6-мерный вектор переменных $z = Ax - b$.
 		
 		Тогда к вектору c дописываем $m$ нулей и рассматриваем вектор
 		$\begin{pmatrix} y\\z \end{pmatrix}$. К матрице справа дописываем единичную матрицу со знаком минус, получаем:
 		\[
 			\begin{cases}
 				(b, 0) \cdot \begin{pmatrix} y\\z \end{pmatrix} \rightarrow min\\
 				\big(A^T(-I)\big) \cdot \begin{pmatrix} y\\z \end{pmatrix} = c\\
 				y, z \geq 0
			\end{cases}
 		\]
 		Двойственная задача не имеет начального углового решения, чтобы его найти необходимо решить вспомогательную задачу. Введем
 		неотрицательный 8-мерный вектор u, тогда получим равенство $Ax +
 		u = b$ и будем решать задачу не на наш минимум (начальный), а на
 		сумму компонент вектора $u$, получим:
 		\[
 			\begin{cases} 
				\sum_{i=1}^{m} u_i \rightarrow min\\
				\big( A^T(-I)I\big) \cdot \begin{pmatrix} y\\z\\u\end{pmatrix} = c\\
				y, z, u \geq 0
			\end{cases}
 		\]
 		И в качестве начальной точки для этой задачи рассмотрим $x = 0$,
 		$а u = b$. Решаем симплекс-методом и если решение $u = 0$, то тогда
 		мы получим точку $x$, для которой $x = b$, $x \geq 0$ и оно допустимое.
 		
 		\textbf{Вспомогательная задача}
 		
 		Составим симплекс-таблицу
 			\begin{widerequation}
 				\scalebox{0.6}{$\begin{pmatrix}
	 				0.0&0.0&0.0&0.0&0.0&0.0&0.0&0.0&0.0&0.0&0.0&0.0&0.0&0.0&1.0&1.0&1.0&1.0&1.0&1.0&0.0\\%
	 				13.0&8.0&15.0&28.0&14.0&1.0&9.0&15.0&-1.0&0.0&0.0&0.0&0.0&0.0&1.0&0.0&0.0&0.0&0.0&0.0&7.0\\%
	 				27.0&16.0&26.0&22.0&11.0&28.0&17.0&19.0&0.0&-1.0&0.0&0.0&0.0&0.0&0.0&1.0&0.0&0.0&0.0&0.0&16.0\\%
	 				9.0&8.0&6.0&4.0&15.0&20.0&6.0&3.0&0.0&0.0&-1.0&0.0&0.0&0.0&0.0&0.0&1.0&0.0&0.0&0.0&5.0\\%
	 				19.0&5.0&17.0&28.0&8.0&23.0&6.0&29.0&0.0&0.0&0.0&-1.0&0.0&0.0&0.0&0.0&0.0&1.0&0.0&0.0&11.0\\%
	 				6.0&13.0&15.0&4.0&15.0&24.0&13.0&5.0&0.0&0.0&0.0&0.0&-1.0&0.0&0.0&0.0&0.0&0.0&1.0&0.0&27.0\\%
	 				27.0&22.0&10.0&17.0&15.0&25.0&23.0&7.0&0.0&0.0&0.0&0.0&0.0&-1.0&0.0&0.0&0.0&0.0&0.0&1.0&10.0%
 				\end{pmatrix}$}
 			\end{widerequation}
 		Выделим базисные столбцы с помощью элементарных преобразований строк. К первой строке добавим все остальные строки, умноженные на $-1$. Получаем:
  			\begin{widerequation}
  				\scalebox{0.6}{$\begin{pmatrix}
  						-101.0&-72.0&-89.0&-103.0&-78.0&-121.0&-74.0&-78.0&1.0&1.0&1.0&1.0&1.0&1.0&0.0&0.0&0.0&0.0&0.0&0.0&-76.0\\%
  						13.0&8.0&15.0&28.0&14.0&1.0&9.0&15.0&-1.0&0.0&0.0&0.0&0.0&0.0&1.0&0.0&0.0&0.0&0.0&0.0&7.0\\%
  						27.0&16.0&26.0&22.0&11.0&28.0&17.0&19.0&0.0&-1.0&0.0&0.0&0.0&0.0&0.0&1.0&0.0&0.0&0.0&0.0&16.0\\%
  						9.0&8.0&6.0&4.0&15.0&\colorbox{blue!30}{20.0}&6.0&3.0&0.0&0.0&-1.0&0.0&0.0&0.0&0.0&0.0&1.0&0.0&0.0&0.0&5.0\\%
  						19.0&5.0&17.0&28.0&8.0&23.0&6.0&29.0&0.0&0.0&0.0&-1.0&0.0&0.0&0.0&0.0&0.0&1.0&0.0&0.0&11.0\\%
  						6.0&13.0&15.0&4.0&15.0&24.0&13.0&5.0&0.0&0.0&0.0&0.0&-1.0&0.0&0.0&0.0&0.0&0.0&1.0&0.0&27.0\\%
  						27.0&22.0&10.0&17.0&15.0&25.0&23.0&7.0&0.0&0.0&0.0&0.0&0.0&-1.0&0.0&0.0&0.0&0.0&0.0&1.0&10.0%
  					\end{pmatrix}$}
  			\end{widerequation}
	 		Разрешающий столбец = 6\\
	 		Разрешающая строка = 4\\
	 		Разрешающий элемент = 20.0
	 		
  			\begin{widerequation}
			\scalebox{0.6}{$\begin{pmatrix}
					-46.55&-23.6&-52.7&-78.8&12.75&0.0&-37.7&-59.85&1.0&1.0&-5.05&1.0&1.0&1.0&0.0&0.0&6.05&0.0&0.0&0.0&-45.75\\%
					12.55&7.6&14.7&27.8&13.25&0.0&8.7&14.85&-1.0&0.0&0.05&0.0&0.0&0.0&1.0&0.0&-0.05&0.0&0.0&0.0&6.75\\%
					14.4&4.8&17.6&16.4&-10.0&0.0&8.6&14.8&0.0&-1.0&1.4&0.0&0.0&0.0&0.0&1.0&-1.4&0.0&0.0&0.0&9.0\\%
					0.45&0.4&0.3&0.2&0.75&1.0&0.3&0.15&0.0&0.0&-0.05&0.0&0.0&0.0&0.0&0.0&0.05&0.0&0.0&0.0&0.25\\%
					8.65&-4.2&10.1&\colorbox{blue!30}{23.4}&-9.25&0.0&-0.9&25.55&0.0&0.0&1.15&-1.0&0.0&0.0&0.0&0.0&-1.15&1.0&0.0&0.0&5.25\\%
					-4.8&3.4&7.8&-0.8&-3.0&0.0&5.8&1.4&0.0&0.0&1.2&0.0&-1.0&0.0&0.0&0.0&-1.2&0.0&1.0&0.0&21.0\\%
					15.75&12.0&2.5&12.0&-3.75&0.0&15.5&3.25&0.0&0.0&1.25&0.0&0.0&-1.0&0.0&0.0&-1.25&0.0&0.0&1.0&3.75%
				\end{pmatrix}$}
		\end{widerequation}
			Разрешающий столбец = 4\\
			Разрешающая строка = 5\\
			Разрешающий элемент = 23.4
			\begin{widerequation}
				\scalebox{0.6}{$\begin{pmatrix}
						-17.42&-37.74&-18.69&0.0&-18.4&0.0&-40.73&26.19&1.0&1.0&-1.18&-2.37&1.0&1.0&0.0&0.0&2.18&3.37&0.0&0.0&-28.07\\%
						2.27&12.59&2.7&0.0&24.24&0.0&\colorbox{blue!30}{9.77}&-15.5&-1.0&0.0&-1.32&1.19&0.0&0.0&1.0&0.0&1.32&-1.19&0.0&0.0&0.51\\%
						8.34&7.74&10.52&0.0&-3.52&0.0&9.23&-3.11&0.0&-1.0&0.59&0.7&0.0&0.0&0.0&1.0&-0.59&-0.7&0.0&0.0&5.32\\%
						0.38&0.44&0.21&0.0&0.83&1.0&0.31&-0.07&0.0&0.0&0.06&0.01&0.0&0.0&0.0&0.0&0.06&-0.01&0.0&0.0&0.21\\%
						0.37&-0.18&0.43&1.0&-0.4&0.0&-0.04&1.09&0.0&0.0&0.05&-0.04&0.0&0.0&0.0&0.0&-0.05&0.04&0.0&0.0&0.22\\%
						-4.5&3.26&8.15&0.0&-3.32&0.0&5.77&2.27&0.0&0.0&1.24&-0.03&-1.0&0.0&0.0&0.0&-1.24&0.03&1.0&0.0&21.18\\%
						11.31&14.15&-2.68&0.0&0.99&0.0&15.96&-9.85&0.0&0.0&0.66&0.51&0.0&-1.0&0.0&0.0&-0.66&-0.51&0.0&1.0&1.06%
					\end{pmatrix}$}
			\end{widerequation}
			Разрешающий столбец = 7\\
			Разрешающая строка = 2\\
			Разрешающий элемент = 9.77
			
			\begin{widerequation}
				\scalebox{0.6}{$\begin{pmatrix}
						-7.94&14.75&-7.43&0.0&82.66&0.0&0.0&-38.45&-3.17&1.0&-6.67&2.59&1.0&1.0&4.17&0.0&7.67&-1.59&0.0&0.0&-25.93\\%
						0.23&1.29&0.28&0.0&2.48&0.0&1.0&-1.59&-0.1&0.0&-0.13&0.12&0.0&0.0&0.1&0.0&0.13&-0.12&0.0&0.0&0.05\\%
						6.19&-4.15&7.97&0.0&-26.42&0.0&0.0&11.54&0.94&-1.0&1.84&-0.42&0.0&0.0&-0.94&1.0&-1.84&0.42&0.0&0.0&4.84\\%
						0.3&0.04&0.13&0.0&0.07&1.0&0.0&0.42&0.03&-0.0&-0.02&-0.03&-0.0&-0.0&-0.03&0.0&0.02&0.03&0.0&0.0&0.19\\%
						0.38&-0.13&0.44&1.0&-0.3&0.0&0.0&1.03&-0.0&0.0&0.04&-0.04&0.0&0.0&0.0&0.0&-0.04&0.04&0.0&0.0&0.23\\%
						-5.85&-4.18&6.55&0.0&-17.63&0.0&0.0&11.43&0.59&0.0&2.02&-0.74&-1.0&0.0&-0.59&0.0&-2.02&0.74&1.0&0.0&20.88\\%
						7.6&-6.42&-7.09&0.0&-38.61&0.0&0.0&\colorbox{blue!30}{15.48}&1.63&0.0&2.81&-1.43&0.0&-1.0&-1.63&0.0&-2.81&1.43&0.0&1.0&0.22%
					\end{pmatrix}$}
			\end{widerequation}
			Разрешающий столбец = 8\\
			Разрешающая строка = 7\\
			Разрешающий элемент = 15.48
			
			\begin{widerequation}
				\scalebox{0.6}{$\begin{pmatrix}
						10.94&-1.19&-25.05&0.0&-13.25&0.0&0.0&0.0&0.89&1.0&0.32&-0.96&1.0&-1.48&0.11&0.0&0.68&1.96&0.0&2.48&-25.39\\%
						1.01&0.63&-0.45&0.0&-1.48&0.0&1.0&0.0&0.07&0.0&0.15&-0.02&0.0&-0.1&-0.07&0.0&-0.15&0.02&0.0&0.1&0.08\\%
						0.52&0.63&13.26&0.0&2.37&0.0&0.0&0.0&-0.27&-1.0&-0.26&0.64&0.0&0.75&0.27&1.0&0.26&-0.64&0.0&-0.75&4.67\\%
						0.1&0.21&0.32&0.0&1.11&1.0&0.0&0.0&-0.01&-0.0&-0.09&0.01&-0.0&0.03&0.01&0.0&0.09&-0.01&0.0&-0.03&0.18\\%
						-0.13&0.3&\colorbox{blue!30}{0.91}&1.0&2.27&0.0&0.0&0.0&-0.11&0.0&-0.14&0.06&0.0&0.07&0.11&0.0&0.14&-0.06&0.0&-0.07&0.21\\%
						-11.46&0.56&11.79&0.0&10.88&0.0&0.0&0.0&-0.62&0.0&-0.06&0.32&-1.0&0.74&0.62&0.0&0.06&-0.32&1.0&-0.74&20.71\\%
						0.49&-0.41&-0.46&0.0&-2.49&0.0&0.0&1.0&0.11&0.0&0.18&-0.09&0.0&-0.06&-0.11&0.0&-0.18&0.09&0.0&0.06&0.01%
					\end{pmatrix}$}
			\end{widerequation}
			Разрешающий столбец = 3\\
			Разрешающая строка = 5\\
			Разрешающий элемент = 0.91
			
			\begin{widerequation}
				\scalebox{0.6}{$\begin{pmatrix}
						7.44&6.95&0.0&27.38&48.95&0.0&0.0&0.0&-2.2&1.0&-3.6&0.6&1.0&0.34&3.2&0.0&4.6&0.4&0.0&0.66&-19.59\\%
						0.95&0.78&0.0&0.49&-0.36&0.0&1.0&0.0&0.01&0.0&0.08&0.0&0.0&-0.07&-0.01&0.0&-0.08&-0.0&0.0&0.07&0.18\\%
						2.37&-3.68&0.0&-14.5&-30.56&0.0&0.0&0.0&1.36&-1.0&\colorbox{blue!30}{1.82}&-0.18&0.0&-0.22&-1.36&1.0&-1.82&0.18&0.0&0.22&1.6\\%
						0.14&0.11&0.0&-0.35&0.32&1.0&0.0&0.0&0.03&-0.0&-0.04&-0.01&-0.0&0.0&-0.03&0.0&0.04&0.01&0.0&-0.0&0.11\\%
						-0.14&0.33&1.0&1.09&2.48&0.0&0.0&0.0&-0.12&0.0&-0.16&0.06&0.0&0.07&0.12&0.0&0.16&-0.06&0.0&-0.07&0.23\\%
						-9.82&-3.27&0.0&-12.89&-18.4&0.0&0.0&0.0&0.84&0.0&1.79&-0.42&-1.0&-0.12&-0.84&0.0&-1.79&0.42&1.0&0.12&17.99\\%
						0.43&-0.27&0.0&0.5&-1.36&0.0&0.0&1.0&0.05&0.0&0.11&-0.06&0.0&-0.03&-0.05&0.0&-0.11&0.06&0.0&0.03&0.12%
					\end{pmatrix}$}
			\end{widerequation}
			Разрешающий столбец = 11\\
			Разрешающая строка = 3\\
			Разрешающий элемент = 1.82
			
			\begin{widerequation}
				\scalebox{0.6}{$\begin{pmatrix}
						12.15&-0.34&0.0&-1.36&-11.64&0.0&0.0&0.0&0.5&-0.98&0.0&0.24&1.0&-0.1&0.5&1.98&1.0&0.76&0.0&1.1&-16.41\\%
						0.84&0.95&0.0&1.15&1.03&0.0&1.0&0.0&-0.05&0.05&0.0&0.01&0.0&-0.06&0.05&-0.05&0.0&-0.01&0.0&0.06&0.11\\%
						1.3&-2.02&0.0&-7.97&-16.81&0.0&0.0&0.0&0.75&-0.55&1.0&-0.1&0.0&-0.12&-0.75&0.55&-1.0&0.1&0.0&0.12&0.88\\%
						0.2&0.02&0.0&-0.7&-0.43&1.0&0.0&0.0&0.06&-0.02&0.0&-0.01&0.0&-0.0&-0.06&0.02&0.0&0.01&0.0&0.0&0.15\\%
						0.06&0.01&1.0&-0.16&-0.15&0.0&0.0&0.0&-0.01&-0.09&0.0&0.05&0.0&0.05&0.01&0.09&0.0&-0.05&0.0&-0.05&0.37\\%
						-12.15&0.34&0.0&1.36&11.64&0.0&0.0&0.0&-0.5&0.98&0.0&-0.24&-1.0&0.1&0.5&-0.98&0.0&0.24&1.0&-0.1&16.41\\%
						0.28&-0.04&0.0&1.38&\colorbox{blue!30}{0.49}&0.0&0.0&1.0&-0.03&0.06&0.0&-0.05&0.0&-0.02&0.03&-0.06&0.0&0.05&0.0&0.02&0.02%
					\end{pmatrix}$}
			\end{widerequation}
			Разрешающий столбец = 5\\
			Разрешающая строка = 7\\
			Разрешающий элемент = 0.49
			
				\begin{widerequation}
				\scalebox{0.6}{$\begin{pmatrix}
						18.89&-1.37&0.0&31.37&0.0&0.0&0.0&23.77&-0.29&0.45&0.0&-1.01&1.0&-0.52&1.29&0.55&1.0&2.01&0.0&1.52&-15.85\\%
						0.24&\colorbox{blue!30}{1.04}&0.0&-1.76&0.0&0.0&1.0&-2.11&0.02&-0.08&0.0&0.12&0.0&-0.02&-0.02&0.08&0.0&-0.12&0.0&0.02&0.06\\%
						11.05&-3.51&0.0&39.28&0.0&0.0&0.0&34.32&-0.39&1.52&1.0&-1.91&0.0&-0.74&0.39&-1.52&-1.0&1.91&0.0&0.74&1.69\\%
						0.45&-0.02&0.0&0.5&0.0&1.0&0.0&0.88&0.03&0.03&0.0&-0.06&0.0&-0.02&-0.03&-0.03&0.0&0.06&0.0&0.02&0.17\\%
						0.15&-0.0&1.0&0.26&0.0&0.0&0.0&0.3&-0.02&-0.07&0.0&0.03&0.0&0.05&0.02&0.07&0.0&-0.03&0.0&-0.05&0.38\\%
						-18.89&1.37&0.0&-31.37&0.0&0.0&0.0&-23.77&0.29&-0.45&0.0&1.01&-1.0&0.52&-0.29&0.45&0.0&-1.01&1.0&-0.52&15.85\\%
						0.58&-0.09&0.0&2.81&1.0&0.0&0.0&2.04&-0.07&0.12&0.0&-0.11&0.0&-0.04&0.07&-0.12&0.0&0.11&0.0&0.04&0.05%
					\end{pmatrix}$}
			\end{widerequation}
			Разрешающий столбец = 2\\
			Разрешающая строка = 2\\
			Разрешающий элемент = 1.04
			
			\begin{widerequation}
				\scalebox{0.6}{$\begin{pmatrix}
						19.21&0.0&0.0&29.04&0.0&0.0&1.32&20.98&-0.26&0.34&0.0&-0.85&1.0&-0.55&1.26&0.66&1.0&1.85&0.0&1.55&-15.78\\%
						0.23&1.0&0.0&-1.69&0.0&0.0&0.96&-2.04&0.02&-0.08&0.0&\colorbox{blue!30}{0.12}&0.0&-0.02&-0.02&0.08&0.0&-0.12&0.0&0.02&0.05\\%
						11.86&0.0&0.0&33.34&0.0&0.0&3.39&27.17&-0.33&1.25&1.0&-1.49&0.0&-0.81&0.33&-1.25&-1.0&1.49&0.0&0.81&1.88\\%
						0.45&0.0&0.0&0.47&0.0&1.0&0.02&0.84&0.03&0.03&0.0&-0.06&0.0&-0.02&-0.03&-0.03&0.0&0.06&0.0&0.02&0.17\\%
						0.15&0.0&1.0&0.25&0.0&0.0&0.0&0.29&-0.02&-0.07&0.0&0.03&0.0&0.05&0.02&0.07&0.0&-0.03&0.0&-0.05&0.38\\%
						-19.21&0.0&0.0&-29.04&0.0&0.0&-1.32&-20.98&0.26&-0.34&0.0&0.85&-1.0&0.55&-0.26&0.34&0.0&-0.85&1.0&-0.55&15.78\\%
						0.6&0.0&0.0&2.66&1.0&0.0&0.09&1.86&-0.07&0.12&0.0&-0.1&0.0&-0.04&0.07&-0.12&0.0&0.1&0.0&0.04&0.05%
					\end{pmatrix}$}
			\end{widerequation}
			Разрешающий столбец = 12\\
			Разрешающая строка = 2\\
			Разрешающий элемент = 0.12
			
			\begin{widerequation}
				\scalebox{0.6}{$\begin{pmatrix}
						20.89&7.19&0.0&16.87&0.0&0.0&8.26&6.32&-0.14&-0.22&0.0&0.0&1.0&-0.7&1.14&1.22&1.0&1.0&0.0&1.7&-15.38\\%
						1.96&8.44&0.0&-14.3&0.0&0.0&8.14&-17.21&0.14&-0.67&0.0&1.0&0.0&-0.18&-0.14&0.67&0.0&-1.0&0.0&0.18&0.46\\%
						14.79&12.58&0.0&12.03&0.0&0.0&15.52&1.52&-0.12&0.25&1.0&0.0&0.0&-1.07&0.12&-0.25&-1.0&0.0&0.0&1.07&2.57\\%
						0.57&0.49&0.0&-0.37&0.0&1.0&0.49&-0.17&0.04&-0.01&0.0&0.0&0.0&-0.03&-0.04&0.01&0.0&0.0&0.0&0.03&0.2\\%
						0.09&-0.26&1.0&0.7&0.0&0.0&-0.25&0.83&-0.02&-0.05&0.0&0.0&0.0&\colorbox{blue!30}{0.05}&0.02&0.05&0.0&0.0&0.0&-0.05&0.36\\%
						-20.89&-7.19&0.0&-16.87&0.0&0.0&-8.26&-6.32&0.14&0.22&0.0&0.0&-1.0&0.7&-0.14&-0.22&0.0&0.0&1.0&-0.7&15.38\\%
						0.79&0.82&0.0&1.28&1.0&0.0&0.87&0.19&-0.05&0.05&0.0&0.0&0.0&-0.06&0.05&-0.05&0.0&0.0&0.0&0.06&0.1%
					\end{pmatrix}$}
			\end{widerequation}
			Разрешающий столбец = 14\\
			Разрешающая строка = 5\\
			Разрешающий элемент = 0.05
			
			\begin{widerequation}
				\scalebox{0.6}{$\begin{pmatrix}
						22.07&3.76&13.05&26.0&0.0&0.0&5.01&17.15&-0.41&-0.84&0.0&0.0&1.0&0.0&1.41&1.84&1.0&1.0&0.0&1.0&-10.65\\%
						2.26&7.58&3.29&-12.0&0.0&0.0&7.33&-14.48&0.08&-0.82&0.0&1.0&0.0&0.0&-0.08&0.82&0.0&-1.0&0.0&0.0&1.65\\%
						16.6&7.33&19.96&26.0&0.0&0.0&10.55&18.09&-0.52&-0.7&1.0&0.0&0.0&0.0&0.52&0.7&-1.0&0.0&0.0&0.0&9.8\\%
						0.62&0.36&0.52&-0.0&0.0&1.0&0.36&0.27&0.03&-0.04&0.0&0.0&0.0&0.0&-0.03&0.04&0.0&0.0&0.0&0.0&0.39\\%
						1.69&-4.89&18.57&13.0&0.0&0.0&-4.63&15.41&-0.38&-0.88&0.0&0.0&0.0&1.0&0.38&0.88&0.0&0.0&0.0&-1.0&6.73\\%
						-22.07&-3.76&-13.05&-26.0&0.0&0.0&-5.01&-17.15&0.41&\colorbox{blue!30}{0.84}&0.0&0.0&-1.0&0.0&-0.41&-0.84&0.0&0.0&1.0&0.0&10.65\\%
						0.88&0.55&1.03&2.0&1.0&0.0&0.62&1.05&-0.07&0.0&0.0&0.0&0.0&0.0&0.07&-0.0&0.0&0.0&0.0&0.0&0.47%
					\end{pmatrix}$}
			\end{widerequation}
			Разрешающий столбец = 10\\
			Разрешающая строка = 6\\
			Разрешающий элемент = 0.84
			
			\begin{widerequation}
				\scalebox{0.6}{$\begin{pmatrix}
						0.0&0.0&0.0&0.0&0.0&0.0&0.0&0.0&0.0&0.0&0.0&0.0&0.0&0.0&1.0&1.0&1.0&1.0&1.0&1.0&0.0\\%
						-19.33&3.9&-9.48&-37.43&0.0&0.0&2.43&-31.26&0.48&0.0&0.0&1.0&-0.98&0.0&-0.48&0.0&0.0&-1.0&0.98&0.0&12.07\\%
						-1.62&4.23&9.19&4.54&0.0&0.0&6.41&3.93&-0.19&0.0&1.0&0.0&-0.83&0.0&0.19&0.0&-1.0&0.0&0.83&0.0&18.6\\%
						-0.35&0.19&-0.05&-1.13&0.0&1.0&0.15&-0.48&0.05&0.0&0.0&0.0&-0.04&0.0&-0.05&0.0&0.0&0.0&0.04&0.0&0.85\\%
						-21.35&-8.81&4.95&-14.13&0.0&0.0&-9.85&-2.48&0.05&0.0&0.0&0.0&-1.04&1.0&-0.05&0.0&0.0&0.0&1.04&-1.0&17.85\\%
						-26.2&-4.46&-15.49&-30.86&0.0&0.0&-5.94&-20.36&0.49&1.0&0.0&0.0&-1.19&0.0&-0.49&-1.0&0.0&0.0&1.19&0.0&12.64\\%
						0.95&0.56&1.07&2.08&1.0&0.0&0.63&1.11&-0.07&0.0&0.0&0.0&0.0&0.0&0.07&0.0&0.0&0.0&-0.0&0.0&0.44%
					\end{pmatrix}$}
			\end{widerequation}
		
		В первой строке не осталось отрицательных элементов (не считая
		значение целевой функции) и $u = 0$, значит найдено оптимальное
		решение для вспомогательной задачи, но начальное угловое и допустимое решение для исходной двойственной задачи.
		
		\underline{Оптимальное решение:}
		\[ \begin{pmatrix} 0 & 0 & 0 & 0 & 0.44 & 0.85 & 0 & 0 & 0 & 12.84 & 18.6 & 12.07 & 0 & 17.85 \end{pmatrix} \]
		
		\textbf{Решение двойственной задачи}
		
		Составим симплекс-таблицу для двойственной задачи. Из прошлой матрицы убираем столбцы, соответствующие вектору $u$, первую
		строку заменяем на расширенный вектор $b$ и значение целевой функции приравниваем к нулю.
		
		\begin{widerequation}
			\scalebox{0.6}{$\begin{pmatrix}
					6.0&5.0&2.0&17.0&2.0&9.0&21.0&5.0&0.0&0.0&0.0&0.0&-0.0&0.0&-0.0\\%
					-19.33&3.9&-9.48&-37.43&0.0&0.0&2.43&-31.26&0.48&0.0&0.0&1.0&-0.98&0.0&12.07\\%
					-1.62&4.23&9.19&4.54&0.0&0.0&6.41&3.93&-0.19&0.0&1.0&0.0&-0.83&0.0&18.6\\%
					-0.35&0.19&-0.05&-1.13&0.0&1.0&0.15&-0.48&0.05&0.0&0.0&0.0&-0.04&0.0&0.85\\%
					-21.35&-8.81&4.95&-14.13&0.0&0.0&-9.85&-2.48&0.05&0.0&0.0&0.0&-1.04&1.0&17.85\\%
					-26.2&-4.46&-15.49&-30.86&0.0&0.0&-5.94&-20.36&0.49&1.0&0.0&0.0&-1.19&0.0&12.64\\%
					0.95&0.56&1.07&2.08&1.0&0.0&0.63&1.11&-0.07&0.0&0.0&0.0&0.0&0.0&0.44%
				\end{pmatrix}$}
		\end{widerequation}
		
		
		Выделяем базисные столбцы с помощью элементарных преобразований строк матрицы.
		\begin{widerequation}
		\scalebox{0.6}{$\begin{pmatrix}
				7.21&2.15&0.27&23.04&0.0&0.0&18.42&7.13&-0.27&0.0&0.0&0.0&0.39&0.0&-8.53\\%
				-19.33&3.9&-9.48&-37.43&0.0&0.0&2.43&-31.26&0.48&0.0&0.0&1.0&-0.98&0.0&12.07\\%
				-1.62&4.23&9.19&4.54&0.0&0.0&6.41&3.93&-0.19&0.0&1.0&0.0&-0.83&0.0&18.6\\%
				-0.35&0.19&-0.05&-1.13&0.0&1.0&0.15&-0.48&\colorbox{blue!30}{0.05}&0.0&0.0&0.0&-0.04&0.0&0.85\\%
				-21.35&-8.81&4.95&-14.13&0.0&0.0&-9.85&-2.48&0.05&0.0&0.0&0.0&-1.04&1.0&17.85\\%
				-26.2&-4.46&-15.49&-30.86&0.0&0.0&-5.94&-20.36&0.49&1.0&0.0&0.0&-1.19&0.0&12.64\\%
				0.95&0.56&1.07&2.08&1.0&0.0&0.63&1.11&-0.07&0.0&0.0&0.0&0.0&0.0&0.44%
			\end{pmatrix}$}
	\end{widerequation}
	Разрешающий столбец = 9\\
	Разрешающая строка = 4\\
	Разрешающий элемент = 0.05
	
	\begin{widerequation}
		\scalebox{0.6}{$\begin{pmatrix}
			5.2&3.27&0.0&16.47&0.0&5.8&19.27&4.33&0.0&0.0&0.0&0.0&0.13&0.0&-3.6\\%
			-15.8&1.93&-9.0&-25.87&0.0&-10.2&0.93&-26.33&0.0&0.0&0.0&1.0&-0.53&0.0&3.4\\%
			-3.0&5.0&9.0&0.0&0.0&4.0&7.0&2.0&0.0&0.0&1.0&0.0&-1.0&0.0&22.0\\%
			-7.4&4.13&-1.0&-24.27&0.0&21.4&3.13&-10.33&1.0&0.0&0.0&0.0&-0.93&0.0&18.2\\%
			-21.0&-9.0&5.0&-13.0&0.0&-1.0&-10.0&-2.0&0.0&0.0&0.0&0.0&-1.0&1.0&17.0\\%
			-22.6&-6.47&-15.0&-19.07&0.0&-10.4&-7.47&-15.33&0.0&1.0&0.0&0.0&-0.73&0.0&3.8\\%
			0.4&0.87&1.0&0.27&1.0&1.6&0.87&0.33&0.0&0.0&0.0&0.0&-0.07&0.0&1.8%
			\end{pmatrix}$}
	\end{widerequation}
		В первой строке не осталось отрицательных элементов, значит
		найдено оптимальное решение.
		
		\underline{Оптимальное решение:}
		\[ \begin{pmatrix} 0 & 0 & 0 & 0 & 1.8 & 0 & 0 & 0 \end{pmatrix} \]
		\underline{Целевая функция:} = 3.6
		
	\section{Приложение}  
	Код программы:		
 	\begin{lstlisting}[language=python]
import math
import numpy as np
import enum

class TaskType(enum.Enum):
	direct = 0
	secondary = 1
	dual = 2

def build_matrix_for_direct_task(a, b, c) -> np.ndarray:
	a_c = np.append([-c], a, axis=0)
	eye = np.append([np.zeros(8,)], np.eye(8), axis=0)
	a_c_eye = np.append(a_c, eye, axis=1)
	return np.append(a_c_eye, np.append([[0]], b, axis=0), axis=1)


def build_matrix_for_secondary_task(a, c) -> np.ndarray:
	a = np.append([np.zeros((8,))], a, axis=0)
	minus_i = np.append([np.zeros((6,))], -np.eye(6), axis=0)
	i = np.append([np.ones((6,))], np.eye(6), axis=0)
	c = np.append([[0]], c.reshape((6, 1)), axis=0)
	res = np.hstack([a, minus_i, i, c])
	for i in range(1, len(res)):
	res[0] = res[0] - res[i]
	return res


def build_matrix_for_dual_task(simplex_table: np.ndarray, b) -> np.ndarray:
	base_col = [False for i in range(len(simplex_table[0]))]
	for j in range(len(simplex_table[0])):
		if not math.isclose(simplex_table[0][j], 0, rel_tol=1e-10):
			continue
		is_base_col = True
		for i in range(len(simplex_table)):
			if not (math.isclose(simplex_table[i][j], 0, rel_tol=1e-10) or math.isclose(simplex_table[i][j], 1, rel_tol=1e-10)):
				is_base_col = False
				break
		base_col[j] = is_base_col
	
		simplex_table = np.delete(simplex_table, np.s_[14:20], axis=1)
		for i in range(len(b)):
			simplex_table[0][i] = b[i]
	
	for j in range(len(simplex_table[0])):
		if not base_col[j]:
			continue
		for i in range(len(simplex_table)):
			if not math.isclose(simplex_table[i][j], 1, rel_tol=1e-10):
				continue
			simplex_table[0] -= simplex_table[i] * simplex_table[0][j]
			break
	
	return simplex_table


def simplex(a, b, c, task_type: TaskType):
	if task_type == TaskType.direct:
		simplex_table = build_matrix_for_direct_task(a, b, c)
		y_bias = 0
	elif task_type == TaskType.secondary:
		simplex_table = build_matrix_for_secondary_task(a, c)
		y_bias = 0
	else:
		simplex_table = build_matrix_for_dual_task(a, b)
		y_bias = 1

	it = 0
	while True:
		it += 1
		resolving_column = None
		min_c = 0
		for i in range(len(simplex_table[0])-1):
			if round(simplex_table[0][i], 13) < min_c:
			min_c = simplex_table[0][i]
		resolving_column = i

	if resolving_column is None:
		break

	min_b = float('inf')
	resolving_stroke = None
	for i in range(len(a) - y_bias):
		if 0 < simplex_table[i+1][-1] / simplex_table[i+1][resolving_column] < min_b:
			min_b = simplex_table[i+1][-1] / simplex_table[i+1][resolving_column]
	resolving_stroke = i + 1

	resolving_element = simplex_table[resolving_stroke][resolving_column]

	simplex_table[resolving_stroke] /= resolving_element

	for i in range(len(simplex_table)):
		if i == resolving_stroke:
			continue
		simplex_table[i] -= simplex_table[i][resolving_column] * simplex_table[resolving_stroke]

	ans = np.zeros(len(simplex_table[0]))
		for j in range(len(simplex_table[0])):
			for i in range(len(simplex_table)):
				if round(simplex_table[i][j], 2) == 1.:
					ans[j] = simplex_table[i][-1]
					break

	return (ans, simplex_table)
 	\end{lstlisting}
		  
\end{document}	
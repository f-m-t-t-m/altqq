\documentclass[a4paper, 14pt]{extarticle}
\usepackage{fefutitle}
\usepackage{listings}
\usepackage{xcolor}
\usepackage{amsmath}
\usepackage{amssymb}
\usepackage{changepage}
\usepackage{makecell}
\usepackage{longtable}

\newenvironment{widerequation}{%
	\begin{adjustwidth}{-2cm}{-2cm}\begin{equation}}
		{\end{equation}\end{adjustwidth}}


\lstdefinestyle{mystyle}{
	basicstyle={\small\ttfamily},
	keywordstyle=\color{orange},
	stringstyle=\color{green},
	basicstyle=\ttfamily\footnotesize,
	breakatwhitespace=false,         
	breaklines=true,                 
	captionpos=b,                    
	keepspaces=true,                 
	numbers=none,                    
	numbersep=5pt,                  
	showspaces=false,                
	showstringspaces=false,
	showtabs=false,                  
	tabsize=2,
	aboveskip=3mm,
	belowskip=3mm,
}
\lstset{style=mystyle}

\begin{document}
	\fefutitle{2}
	\pagebreak	

	\section{Постановка задачи}
		Найти минимум функции $R^n$: 
		\[ f(x) = \dfrac{1}{2} \cdot x^TAx + b \cdot x \]
		с условием $\|x - x_0\| \leq r$
		
		\textbf{Исходные данные:}
		
		\hphantom{--}$A$ -- произвольная симметрическая невырожденная матрица, $A \in \mathbb{R}^{4}$
		
		\hphantom{--}$b$	-- произвольный ненулевой вектор, $b \in \mathbb{R}^{4}$ 
		
		\hphantom{--}$x_0$ -- произвольный начальный ненулевой вектор, $ x \in\mathbb{R}^{4} $
		\begin{widerequation}
			A = 
			\begin{pmatrix}
				0.99424 & 1.07004 & 1.13757 & 1.11400\\
				0.90153 & 0.96192 & 1.01808 & 0.99415\\
				1.13121 & 1.20908 & 1.27350 & 1.26081\\
				1.17758 & 1.25204 & 1.32679 & 1.30954
			\end{pmatrix}
			b = 
			\begin{pmatrix} 
				1.82801\\
				1.89756\\
				1.52039\\
				1.29904
			\end{pmatrix}
			x_0 = 
			\begin{pmatrix} 
				1.29117\\
				1.52013\\
				1.98189\\
				1.44639
			\end{pmatrix}	
			r = 5
		\end{widerequation}
 	Алгоритм содержится в приложении 1.
 	\section{Решение}
 		Найдем функцию Лагранжа:
 		\[ L(x,y) = \dfrac{1}{2}x^TAx + bx + y\Bigl(\|x-x_0\|^2 - r^2\Bigr) \]	
 		
 		Найдем точки минимума. Для этого возьмем частную производную по $x$ и приравняем ее к нулю:
 		\[ \dfrac{\partial L}{\partial x} = Ax + b + 2y(x - x_0) = 0 \]
 		\pagebreak
 		
 		Рассмотрим два случая:
 		\begin{enumerate}
 			\item Пусть $y = 0$
 			
 			$Ax + b = 0$, тогда $x_* = -A^{-1}b$, где $x_*$ -- <<подозрительная>> на минимум точка
 			\[ 
	 			x_* = 
	 			\begin{pmatrix}
	 				\hphantom{-}23.24602137\\
	 				-77.78375858\\
	 				-18.34034689\\
	 				\hphantom{-}71.05482568
	 			\end{pmatrix}
 				f(x_*) = -20.34363927
 			\]
 			Проверим, подходит ли данная точка под условие $\|x - x_0\| \leq r$:
 			\[ 
 				\begin{Vmatrix}
 					21.95484357\\
 					-79.30388908\\
 					-20.32224649\\
 					 69.60843178
				\end{Vmatrix}
				=
				109.67884689 > r
			\]
			Условие не выполняется. Таким образом, найденная точка не подходит под ограничения и не будет рассматриваться при выборе итогового ответа.
			
			\item Пусть $y > 0$
			
			Преобразуем $L'_x$ и получим следующую систему из пяти уравнений:
			\[ 
				\begin{cases}
					(A + 2Iy)x + (b + 2yx_0) = 0,\\
					\|x - x_0\|^2 - r^2 = 0.
				\end{cases}			
			 \]
			 Для нахождения точек, подозрительных на оптимум, воспользуемся методом	Ньютона:
			 \[ x_{k+1} = x_k - f'(x_k)^{-1} \cdot f(x_k), \]
			 где $x_k$ -- пятимерный вектор неизвестных, составленный из элементов $x$ и $y$
			 
			 $f(x_k)$ -- левая часть, данной системы,
			 
			 $f'(x_k)$ -- матрица Якоби данной системы уравнения.
			 \[ 
				 f'(x) = J = 
				 \begin{pmatrix}
				 	A + 2Iy & 2(x-x_0)\\
				 	2(x-x_0)^T & 0
				 \end{pmatrix} 
		 	\]
 		\end{enumerate}
	 	Метод Ньютона будем запускать на нескольких начальных приближений, т.к. функция может иметь несколько оптимальных точек:
	 	\begin{widerequation}
	 		x_1 = 
	 		\begin{pmatrix}
	 			-2.7088222\\
	 			\hphantom{-}1.5201305\\
	 			\hphantom{-}1.9818996\\
	 			\hphantom{-}1.4463939 
	 		\end{pmatrix}
	 		x_2 = 
	 		\begin{pmatrix}
	 			5.2911778\\
	 			1.5201305\\
	 			1.9818996\\
	 			1.4463939
	 		\end{pmatrix}
	 		x_3 = \begin{pmatrix}
	 			\hphantom{-}1.2911778\\
	 			-2.4798695\\
	 			\hphantom{-}1.9818996\\
	 			\hphantom{-}1.4463939
	 		\end{pmatrix}
	 		x_4 = \begin{pmatrix}
	 			1.2911778\\
	 			5.5201305\\
	 			1.9818996\\
	 			1.4463939
	 		\end{pmatrix}
	 	\end{widerequation}
	 	\begin{widerequation}
	 		x_5 = 
	 		\begin{pmatrix}
	 			\vphantom{-}1.2911778\\
	 			\vphantom{-}1.5201305\\
	 			-2.0181004\\
	 			\vphantom{-}1.4463939
	 		\end{pmatrix}
	 		x_6 = 
	 		\begin{pmatrix}
	 			1.2911778\\
	 			1.5201305\\
	 			5.9818996\\
	 			1.4463939\\
	 		\end{pmatrix}
	 		x_7 = \begin{pmatrix}
	 			\vphantom{-}1.2911778\\
	 			\vphantom{-}1.5201305\\
	 			\vphantom{-}1.9818996\\
	 			-2.5536061
	 		\end{pmatrix}
	 		x_8 = \begin{pmatrix}
	 			1.2911778\\
	 			1.5201305\\
	 			1.9818996\\
	 			5.4463939
	 		\end{pmatrix}
	 	\end{widerequation}
	 	Условие выхода из цикла:
	 	\[ \|x_{k+1} - x_{k}\| \leq \varepsilon, \]
	 	где $\varepsilon = 10^{-6}$
	 	
	 	В результате получаем несколько точек $x_i$, подозрительных на оптимум:
	 	\begin{longtable}[c]{ |c|c|c|c|c|}
	 		\hline
	 		i & \makecell{Начальное\\приближение} & $x_i$ & $y_i$ & $f(x_i)$\\
	 		\hline
	 		1 & $\begin{pmatrix} 
		 			-2.7088222\\
					\hphantom{-}1.5201305\\
		 			\hphantom{-}1.9818996\\
		 			\hphantom{-}1.4463939\\
		 			\hphantom{-}4  
 				\end{pmatrix}$ 
 				& $\begin{pmatrix} 
 					-0.73928961\\
 					-0.99715835\\
 					-1.18075373\\
 					-0.68388375 
 				\end{pmatrix}$ 
 				& $2.2349577$ & $1.434714$\\
	 		\hline
	 		2 & $\begin{pmatrix} 
	 			5.2911778\\
	 			1.5201305\\
	 			1.9818996\\
	 			1.4463939\\
	 			\hphantom{-}4  
	 		\end{pmatrix}$ 
	 		& $\begin{pmatrix} 
	 			-0.73928961\\
	 			-0.99715835\\
	 			-1.18075373\\
	 			-0.68388375 
	 		\end{pmatrix}$ 
	 		& $-2.578362$ & $1.434714$\\
	 		\hline
	 		3 & $\begin{pmatrix} 
	 			\hphantom{-}1.2911778\\
	 			-2.4798695\\
	 			\hphantom{-}1.9818996\\
	 			\hphantom{-}1.4463939\\
	 			\hphantom{-}4  
	 		\end{pmatrix}$ 
	 		& $\begin{pmatrix} 
	 			-0.73928961\\
	 			-0.99715835\\
	 			-1.18075373\\
	 			-0.68388375 
	 		\end{pmatrix}$ 
	 		& $2.877857$ & $1.434714$\\
	 		\hline
	 		4 & $\begin{pmatrix} 
	 			\hphantom{-}1.2911778\\
	 			-2.4798695\\
	 			\hphantom{-}1.9818996\\
	 			\hphantom{-}1.4463939\\
	 			\hphantom{-}4  
	 		\end{pmatrix}$ 
	 		& $\begin{pmatrix} 
	 			-0.73928961\\
	 			-0.99715835\\
	 			-1.18075373\\
	 			-0.68388375 
	 		\end{pmatrix}$ 
	 		& $-3.457333$ & $1.434714$\\
	 		\hline
	 		5 & $\begin{pmatrix} 
	 			\vphantom{-}1.2911778\\
	 			\vphantom{-}1.5201305\\
	 			-2.0181004\\
	 			\vphantom{-}1.4463939\\
	 			\hphantom{-}4  
	 		\end{pmatrix}$ 
	 		& $\begin{pmatrix} 
	 			-0.73928961\\
	 			-0.99715835\\
	 			-1.18075373\\
	 			-0.68388375 
	 		\end{pmatrix}$ 
	 		& $3.281529$ & $1.434714$\\
	 		\hline
	 		6 & $\begin{pmatrix} 
	 			1.2911778\\
	 			1.5201305\\
	 			5.9818996\\
	 			1.4463939\\
	 			\hphantom{-}4  
	 		\end{pmatrix}$ 
	 		& $\begin{pmatrix} 
	 			-0.73928961\\
	 			-0.99715835\\
	 			-1.18075373\\
	 			-0.68388375 
	 		\end{pmatrix}$ 
	 		& $-4.974945$ & $1.434714$\\
	 		\hline
	 		7 & $\begin{pmatrix} 
	 			\vphantom{-}1.2911778\\
	 			\vphantom{-}1.5201305\\
	 			\vphantom{-}1.9818996\\
	 			-2.5536061\\
	 			\hphantom{-}4  
	 		\end{pmatrix}$ 
	 		& $\begin{pmatrix} 
	 			-0.73928961\\
	 			-0.99715835\\
	 			-1.18075373\\
	 			-0.68388375 
	 		\end{pmatrix}$ 
	 		& $2.244119$ & $1.434714$\\
	 		\hline
	 		8 & $\begin{pmatrix} 
	 			\vphantom{-}1.2911778\\
	 			\vphantom{-}1.5201305\\
	 			\vphantom{-}1.9818996\\
	 			-2.5536061\\
	 			\hphantom{-}4  
	 		\end{pmatrix}$ 
	 		& $\begin{pmatrix} 
	 			1.2911778\\
	 			1.5201305\\
	 			1.9818996\\
	 			5.4463939
	 		\end{pmatrix}$ 
	 		& $-2.869787$ & $1.434714$\\
	 		\hline
	 	\end{longtable}
	 	Выясниим, в какой из данных точек функция принимает минимальное значение. Отбросим результаты, полученные при $y < 0$, и получим, что минимальное значение функции $f(x)$ при заданных ограничениях достигается в точке:
	 	\[
	 		x_{min} = 
	 		\begin{pmatrix}
	 			-0.73928961\\
	 			-0.99715835\\
	 			-1.18075373\\
	 			-0.68388375
	 		\end{pmatrix}
	 	 \]
	 	 Минимальное значение функции:
	 	 \[ f_{min}(x) = 1.434714 \]
 	\section{Приложение}
 	\begin{lstlisting}[language=python]
import numpy as np
		
		
def f(x: np.ndarray) -> float:
	res = .5*x.transpose()@A@x + b.transpose()@x
	return res[0][0]


def lagrange_slae(x: np.ndarray) -> np.ndarray:
	return np.append((A + 2*np.eye(4)*y)@x + (b + 2*y*x_0), [[np.linalg.norm(x - x_0)**2 - r**2]], axis=0)


def jacobian(x: np.ndarray) -> np.ndarray:
	J_1_1 = A + 2*np.eye(4)*y
	J_1_2 = 2*(x - x_0)
	J_2_1 = J_1_2.transpose()
	J_2_2 = [[0]]
	J_1 = np.append(J_1_1, J_1_2, axis=1)
	J_2 = np.append(J_2_1, J_2_2, axis=1)
	return np.append(J_1, J_2, axis=0)


def newton(x_k: np.ndarray, epsilon=1e-6, max_iter=30):
	x_prev = x_k
	x_cur = x_prev - np.linalg.inv(jacobian(x_prev[0:-1])) @ lagrange_slae(x_prev[0:-1])
	it = 0
	while np.linalg.norm(x_cur[0:-1] - x_prev[0:-1]) > epsilon and it < max_iter:
		it += 1
		x_prev = x_cur
		x_cur = x_prev - np.linalg.inv(jacobian(x_prev[0:-1])) @ lagrange_slae(x_prev[0:-1])
	return x_cur


if __name__ == '__main__':
	A = np.loadtxt("a.txt", usecols=(range(4)))
	b = np.loadtxt("b.txt", usecols=(range(1)), ndmin=2)
	x_0 = np.loadtxt("x_0.txt", usecols=(range(1)), ndmin=2)
	r = 5
	a = 4
	y = 3
	sign = 1
	x_ = np.append(x_0, [[y]], axis=0)

x_star = -np.linalg.inv(A) @ b
f_in_x_star = f(x_star)

for i in range(8):
	sign = -sign
	x_k = x_.copy()
	x_k[i//2][0] += sign * a
	res = newton(x_k)
 	\end{lstlisting}
		  
\end{document}	